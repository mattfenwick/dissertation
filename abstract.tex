\pagestyle{myabstract} % turn on the new page style
\thispagestyle{empty} % but turn it off for the first page

\begin{center}
  {\LARGE \mattftitle{}}
  
  \vspace{2cm}

  Matthew Fenwick, PhD
  
  \vspace{0.5cm}

  University of Connecticut, 2014
  
  \vspace{1in}
\end{center}

\thispagestyle{empty}
Nuclear Magnetic Resonance (NMR) spectroscopy is a technique for studying 
biological molecules such as proteins at the atomic level.  
The information obtained from NMR is used to identify 
binding partners, locate active sites and binding pockets, and obtain 
structural and dynamics information which can be used in drug design.  In 
order to study molecules using NMR, an NMR spectrometer is used to collect 
free-induction decay (FID) data sets from a pure, high-concentration sample 
of the molecule(s) of interest.  In subsequent analysis, the FID data is 
processed to frequency-domain spectra, which are then analysed to find peaks 
and assign the peaks to specific atoms in the molecule, in a process known as 
chemical shift assignment.  The typical process makes use of automated tools to 
speed up simple and tedious tasks where possible, but relies upon manual 
analysis for complicated and difficult cases.  Spectroscopists use a deductive 
strategy of iteratively applying previously identified rules to make analyses 
of specific cases.  Ambiguous cases are noted and deferred, or the highest 
probability interpretation is made.  Following chemical shift assignment, 
NOESY spectra are peak-picked and assigned, and finally a structure is 
calculated and refined.  During the analysis process, large amounts of data 
and metadata are generated.  However, much of this is not recorded and thus 
does not show up in archives such as the BMRB.  This raises serious 
reproducibility concerns, since the data and metadata describing how the 
analysis was carried out are lost.  These concerns lead to practical issues,
including how to collaborate when data is missing, how to efficiently identify 
and correct errors, and how can to augment analysis with additional data  
without having to restart the process from the beginning.
The growing problems caused by irreproducibility in science have been noted 
recently.  The main contribution of this project is a definition of 
reproducibility within protein NMR, a strategy for rendering NMR analysis 
reproducible, a software implementation to enable reproducible analysis, a 
means for sharing reproducible data sets through a public archive; and a data 
set analysed using fully reproducible means.

\clearpage
\pagestyle{plain}

