\documentclass[a4paper]{report}
\usepackage[top=30mm, bottom=30mm, left=30mm, right=30mm]{geometry}
\usepackage{times} % ???
\usepackage{graphicx} % for pictures
\usepackage{hyperref} % for internal links
\linespread{2}
\DeclareGraphicsExtensions{.jpg}
\usepackage[T1]{fontenc} % see http://tex.stackexchange.com/questions/664/

\begin{document}

\begin{abstract} % on the abstract page, we also need the title, my name, university of ..., graduation year
Nuclear Magnetic Resonance (NMR) spectroscopy is a technique for studying 
biological molecules such as proteins and metabolites at the atomic level.  
The information obtained from NMR is used to identify metabolites, identify 
binding partners, locate active sites and binding pockets, and obtain 
structural and dynamics information which can be used in drug design.  In 
order to study molecules using NMR, an NMR spectrometer is used to collect 
free-induction decay (FID) data sets from a pure, high-concentration sample 
of the molecule(s) of interest.  In subsequent analysis, the FID data is 
processed to frequency-domain spectra, which are then analysed to find peaks 
and assign the peaks to specific atoms in the molecule, in a process known as 
chemical shift assignment.  The typical process makes use of automated tools to 
speed up simple and tedious tasks where possible, but relies upon manual 
analysis for complicated and difficult cases.  Spectroscopists use a deductive 
strategy of iteratively applying previously identified rules to make analyses 
of specific cases.  Ambiguous cases are noted and deferred, or the highest 
probability interpretation is made.  Following chemical shift assignment, 
NOESY spectra are peak-picked and assigned, and finally a structure is 
calculated and refined.  During the analysis process, large amounts of data 
and metadata are generated.  However, much of this is not recorded and thus 
does not show up in archives such as the BMRB.  This raises serious 
reproducibility concerns, since the data and metadata describing how the 
analysis was carried out are lost.  Additional concerns include: how can 
practitioners successfully collaborate when data is missing?  How can errors 
be efficiently identified and corrected?  How can additional data be used to 
augment the analysis without having to restart the process from the beginning?  
The growing problems caused by irreproducibility in science have been noted 
recently.  The main contribution of this project is a definition of 
reproducibility within protein NMR, a strategy for rendering NMR analysis 
reproducible, a software implementation to enable reproducible analysis, a 
means for sharing reproducible data sets through a public archive; and a data 
set analysed using fully reproducible means.
\end{abstract}

\begin{titlepage}

Reproducible Protein NMR Data Analysis

Matthew Fenwick

B.S., University of Oklahoma, 2009

A Dissertation 
Submitted in Partial Fulfillment of the 
Requirements for the Degree of Doctor of Philosophy 
at the 
University of Connecticut 

2014
\end{titlepage}

\pagenumbering{roman}
% do I need to add the approval page in here ??

\tableofcontents

\listoftables

\listoffigures

\chapter{Introduction}
\pagenumbering{arabic}
\section{Protein NMR}
NMR (Nuclear Magnetic Resonance) spectroscopy is an experimental technique for 
studying proteins and other biological molecules at atomic resolution.  In 
comparison to other techniques for high-resolution characterization of 
biological molecules, NMR’s significance as an experimental technique stems 
from its ability to collect detailed molecular data, from which further 
analysis can derive not only structural information but also dynamics, 
activity, and interactions with other molecules.  The importance of NMR 
spectroscopy to the structural biology community has steadily increased, as 
measured by the number of biologically relevant molecules that have been 
studied using the technique, and the corresponding data deposited into 
publicly available databases -- since 1990, the structures of nearly 9,000 
proteins that were solved using NMR have been deposited in the Protein Data 
Bank (PDB) \cite{pdb}, a facility for the archival and sharing of protein-related 
data, and NMR data is available for over 10,000 proteins in the BioMagnetic 
Resonance Bank (BMRB) \cite{bmrb}, a facility for the archival and sharing of 
specifically NMR-derived data.  The data collected using NMR techniques is 
important to the field of drug design, as it can aid in identifying potential 
binding partners based on surfaces as well as actual binding partners based on 
chemical shifts, and understanding biological processes.

In order to study proteins in solution using NMR, multi-step processes are 
employed to collect and analyze data.  An example process is described here 
as a series of independent stages.  Here is a brief outline of the example 
process, which will be expanded upon later:

\begin{itemize}
  \item data collection
  \begin{itemize}
     \item isolation and purification of sample of interest
     \item collection of time-domain data in an NMR spectrometer
  \end{itemize}
  \item spectral processing
  \begin{itemize}
     \item processing of time-domain data to frequency-domain spectra
  \end{itemize}
  \item spectral analysis
  \begin{itemize}
     \item peak-picking of through-bond frequency-domain spectra
     \item build spin systems by analysis of peak chemical shifts
     \item build sequential spin system chains
     \item amino acid types of spin systems
     \item peaktypes of peaks
     \item resonances to peak dimensions
     \item resonances to atoms
     \item spin systems to residues
  \end{itemize}
  \item structure determination
  \begin{itemize}
    \item peak-picking of NOESY spectra
    \item assignment of resonances to NOESY peaks
    \item structure calculation
    \item stereospecific resonance assignment
    \item structure refinement
  \end{itemize}
\end{itemize}

First, the protein/molecule of interest is isolated and a concentrated solution 
is obtained.  Then, the solution is placed in an NMR spectrometer and an array 
of time-domain free induction decay (FID)s are collected.  These experiments 
exploit the coupling constants and characteristic chemical shifts of specific 
atoms and functional groups in order to correlate chemical shifts of 
covalently-bound atoms.

Second, spectral processing operates on these FID data sets.  They are 
converted to frequency-domain spectra using tools such as NMRPipe \cite{nmrpipe}
and the Rowland NMR ToolKit \cite{rnmrtk}.  Functions such as 
zero-fills, Fourier transforms, phase shifts, apodizations, and linear 
predictions are applied to the data as a processing pipeline.  These 
functions are used to ensure that the spectra are amenable to further 
analysis, by optimizing peak size and shape and minimizing processing 
artifacts.

Third is the spectral analysis stage, in which the goal is to identify the 
chemical shifts of individual atoms by process known as resonance assignment.
The spectra may be analyzed using a tool such as XEasy \cite{xeasy}, 
Sparky \cite{sparky}, NMRViewJ \cite{nmrviewj}, or CCPN Analysis \cite{ccpn}.  
In each spectrum, peak-picking is performed, and true signal peaks must 
be identified and separated from peaks caused by noise and artifacts.  
Additionally, signal peaks caused by contaminants must be identified.  
Next, spin systems are identified and constructed \cite{ccpn}. 
A spin system is a network of covalently-bonded resonances visible through 
overlapping NMR experiments.  Spin systems are composed of resonances; a 
resonance is an NMR-visible signal that corresponds to an atom appearing 
at a specific chemical shift in one or more experiments \cite{ccpn}.  
The connectivity of resonances in a spin system is exploited 
in through-bond experiments.  Spin systems then must be assigned 
connectivities to other spin systems through overlap of mutual resonances, 
amino acid types, and finally specific residues of the sample of interest. 
Resonances must also be assigned to specific atoms \cite{ccpn}, 
with the final result being that specific atoms in the sample of interest 
are assigned chemical shift values.  Currently, 100\% assignments are not 
achievable due to several factors [need reference]: 
\begin{itemize}
  \item data quality
  \item ambiguity
  \item missing resonances due to local dynamics
  \item metal ions
\end{itemize}
However, 90-95\% completion is often sufficient [reference].

In the fourth and final stage, the chemical shift assignments are used to 
interpret the other class of experimental NMR data, Nuclear Overhauser Effect 
spectroscopy (NOESY) experiments.  NOESY experiments use through-space 
transfer of magnetization to identify spatially near pairs of Hydrogen nuclei, 
regardless of the number of chemical bonds between them.  NOESY spectra are 
processed and peak-picked, similarly to through-bond spectra, and resonance 
assignments of peaks made.  The resonance assignments of the NOESY data are 
interpreted to obtain distance restraints, which are then used to calculate 
coarse-grained three-dimensional structures.  The structures may then be 
refined and fine-tuned using a computational tool such as Amber \cite{amber}.  
Unambiguous resonance assignment of NOESY data purely on the basis of chemical 
shift assignments may often be impossible or impractical, due to degenerate 
chemical shifts and to non-stereospecific assignments.  While these 
ambiguities can often be resolved through the collection of additional 
NMR data, the expense involved in doing so may often make it more practical 
to attempt to resolve the ambiguities through a structure determination 
program such as CYANA \cite{cyana2004}.

The massive amount of data involved in a structure determination process --
often on the order of gigabytes -- necessitates the use of computational
tools for data management as well as efficiency of analysis.  To address
specific problems in the NMR analysis process, many software implementations 
of useful data processing algorithms have been created, distributed, and 
maintained in recent years [citation].  Additionally, several groups have 
accelerated the process by producing software tools spanning and integrating 
multiple steps to decrease the necessity for time-consuming human intervention.
This allows automated or semi-automated structure determination [3] for small 
proteins.  Other groups have built integrated pipelines, using one specific 
tool for each step [4 and 5], and allowing manual intervention at traditionally 
difficult stages.  Many recent methods re-envision structure determination 
as an iterative process, where the results of a later stage may require the 
researcher to re-evaluate or re-perform an earlier stage [6]; this has been 
applied to interpretation of NOE-derived restraints [7].  Altogether, the 
structure determination process can often take several months [8].

In general, while computational tools are able to deliver results relatively 
quickly compared to manual analysis, they are not able to produce more 
accurate results, especially in the case of low-quality, irregular, or 
otherwise problematic data, resulting in false positives and false 
negatives.  Examples of analysis stages where automated tools may be 
inadequate include:
\begin{itemize}
 \item peak-picking
 \item spin system construction
 \item sequential spin system assignment
 \item resonance-peak dimension assignment
 \item resonance-atomtype assignment
 \item sequence-specific spin system assignment
\end{itemize}

This has the consequence that NMR structure determination data analysis 
processes cannot be fully automated if high-quality results are required.  
An effective solution to this problem combines the strengths of the automated 
and manual approaches, in a semi-automated fashion:  computational tools are 
used to quickly perform the majority of analyses such as peak-picking and 
spin system construction, and manual analysis is used to clear up the 
relatively small number of cases involving ambiguities and errors caused 
by problematic or unclear data.  Thus, some amount of manual analysis may 
be required at all stages of the data analysis process 
\cite{guntert2009automated, williamson2009automated}.   
Manual analysis follows a general pattern:
\begin{enumerate}
  \item identification: a feature of the data is identified as amenable to 
  interpretation.  For example, the feature may be a false negative (such as 
  a signal peak misclassified as noise by the automated peak-picker), a false 
  positive (such as an artifactual peak misclassified as signal), or an 
  ambiguity (such as overlapped spin systems that a clustering algorithm was 
  unable to separate into two distinct spin systems).
  \item pattern recognition: the spectroscopist identifies a potential method 
  for interpreting the feature based on his/her domain knowledge of NMR and 
  experience with interpretation of previous data sets.  For example, such 
  methods may take the form of deductive rules:  if <the data matches a 
  certain pattern> , then <it could be interpreted a certain way>.
  \item application of the rule to the data feature.  The chosen rule is 
  applied, and the result of the interpretation is included back into the 
  data set.  The result may now be used to drive further deductions.
  \item repeat -- go to step 1 to identify features for further interpretations
This method is a form of iterative, sequential deduction.  The key components 
are the ordered series of steps, the state of the data before and after each 
step, and the deductive rules used to make interpretations at each step.  In 
addition, it should be noted that the final data set can not be regenerated 
using automated tools alone if there are any manual modifications made to 
tool output.
\end{enumerate}


\section{An Overview of Scientific Methods and Reproducibility}
The term "science" can refer to both the enterprise of knowledge acquisition 
through empirical means, and the body of knowledge acquired through such means. 
The core of science is the notion of reproducibility -- that claims can be 
independently tested and verified.  In science, reproducible knowledge can be 
obtained using a set of general techniques collectively known as "the 
scientific method".  These techniques share many characteristics, most notably:
\begin{itemize}
 \item data collection: observation of a natural phenomenon, in which 
 experiments are performed and the results quantified and recorded
 \item analysis, in which the data collected in an experiment is processed to 
 gain information, knowledge, and understanding
 \item experimental design, in which a researcher invents an experiment 
 procedure for data collection which is intended to provide data testing 
 specific variables of a system, preventing other variables from confounding 
 the results
 \item hypotheses, which synthesize and organize the information, knowledge, 
 and understanding gained in order to explain the observed results and predict 
 the results of further observations
\end{itemize}

For example, one way to arrange these components is in an iterative cycle.  
Here is an arbitrary ordering in which the output of each step is the input 
for the following step:
\begin{itemize}
  \item data collection
  \item data analysis
  \item hypothesis
  \item experimental design
  \item data collection (repeat cycle from beginning)
\end{itemize}

Of course, such an ordering does not capture the full relationships between 
the various components.  For example, experiments are designed to test 
specific hypotheses.  However, in general, by iteratively designing, 
performing, and analysing experiments, and using the results to reach new 
conclusions, which lead to new hypotheses, and new experiments, scientists 
are able to build such knowledge.  The conclusions then lead to ideas for 
new theories and experiments.  

A basic principle of scientific methods is that they enable reproducibility.  
In order for scientific results to be reproducible, all components of the 
experimental design, data collection and data analysis steps must be 
reproducible.  These include:
\begin{itemize}
 \item Experimental design.  Our understanding of scientific methods, the 
 significance of reproducibility, and means of achieving reproducibility 
 have evolved along with our experimental methods.  Lab notebooks and journal 
 publications are typical media for enabling reproducibility of experimental 
 designs.  
  \begin{itemize}
     \item is the experimental design recorded in sufficient detail to share 
     it fully with other scientists?
     \item does the experimental design acknowledge the possibility of 
     confounding variables, and take measures to control for them?
     \item given an experimental design, can it be repeatedly executed and 
     identical results obtained (within expected deviations due to experimental 
     error and variables outside the control of the experiment)?
  \end{itemize}
 \item Experimental execution.  Lab notebooks and Library Information Systems 
 (LIMS) are also important for recording the details of experiments, including 
 any problems such as contamination.
  \begin{itemize}
     \item is the actual procedure executed recorded in sufficient detail, 
     including any deviations from the given experimental design?
     \item is the sampling biased?
  \end{itemize}
  \item Computational.  As computers continue to play an ever-growing role in 
  science, scientists have noted the problems that indiscriminate computational 
  use poses for reproducibility \cite{donoho_wavelab, peng2011reproducible}.  
  By the early 
  1990’s, researchers began defining reproducible computational analysis, and 
  describing strategies for achieving it. 
  \begin{itemize}
     \item are all tools, scripts, platforms and code available, including the 
     exact versions used?
     \item were all parameterizations recorded?
     \item was all input and output data recorded?
  \end{itemize}
  \item Analysis.  A further concern of reproducibility is with 
  non-computational analysis; failing to recognize analytical bias and to 
  use appropriate statistical measures has long been a source of 
  irreproducibility in scientific endeavors \cite{sackett1979bias}.
  This has also been previously noted in \cite{ioannidis2005most} 
  and in \cite{nuzzo2014statistical}, 
  and was the focus of a Nature special [reference http://www.nature.com/nature/focus/reproducibility/].
  \begin{itemize}
     \item were all manual changes to data sets recorded?
     \item were necessary and sufficient statistical measures used?
     \item were inappropriate statistical measures used?
     \item were sources of bias recognized and handled?
  \end{itemize}
\end{itemize}

Reproducibility is important to science for several reasons.  First, it 
provides a means for measuring the quality and usefulness of a study or 
claim.  Second, reproducibility facilitates knowledge transfer between 
peers, which enables fellow researchers to build on the foundation provided 
by a study, whether by extending the experimental design and data collection, 
applying the experiment in a different context, or applying additional analyses 
to existing data.  Third, reproducibility promotes collaboration between fellow 
scientists by enabling sharing of data, information, knowledge and experimental 
designs.  Fourth, reproducibility reduces time wastage due to inability to 
replicate results \cite{ioannidis2005most, mullard2011reliability}. 


\section{Reproducible Protein NMR:  Computation and Analysis}
Successful achievement of a reproducible NMR study requires reproducibility at 
each stage of the process.  First, the protocol for expressing, purifying, and 
preparing the sample of interest for experiments inside the NMR spectrometer 
must be reproducible, as well as the exact experimental conditions, 
spectrometer, pulse sequences and collection times used to collect the 
time-domain data must be captured.  Second, the software, platform, functions, 
and parameterizations for spectral processing stage must be captured in full.  
Third, both the computational results of peak-picking, spin system construction,
spin system assignment, and resonance assignment as well as any manual changes, 
along with the associated deductive process of reasoning, must be captured.  
Fourth, analysis and assignment of NOESY spectra, structure calculation, 
stereospecific resonance assignment and structure refinement must be captured.  
This last stage may also include computational as well as manual analysis 
components.  

This work will focus on reproducibility of the third and fourth stages, 
spectral analysis and structure determination.  Unfortunately, according to 
the definition of reproducible NMR given above, these stages are irreproducible 
because of:
\begin{itemize}
  \item uncaptured primary data: extraneous results, such as spin systems from 
  contaminants, or noise peaks comparable in size to signal peaks
  \item uncaptured primary data: the intermediate results -- tool output before 
  manual analysis, and the state of the data before and after deductions during 
  manual analysis
  \item uncaptured meta data:  the deductive reasoning used during manual analysis
  \item undeposited primary and meta data: while contemporary NMR databases, 
  such as the BMRB, archive, persist, and disseminate data and analysis 
  results of completed structural, dynamics, and binding studies, the 
  archived data and analysis do not include full information on spin systems 
  and resonances, nor the meta data of manual analysis.  This is because the 
  data is thrown away during the analysis process and not available for deposition.
  \item undeposited meta data: notes of odd, ambiguous, abnormal, or 
  otherwise unexpected situations noticed during analysis
\end{itemize}

Much work has been done to capture additional data from the assignment process.  
The CCPNMR effort, including the significant projects of CCPN Analysis and the 
CCPN data model \cite{ccpn}, captures includes peaks, atoms, residues, 
resonances, and spin systems, along with other significant NMR data pieces.  
However, there are three shortcomings.  First, intermediate results are not 
captured.  Second, only a subset of these data are deposited into the BMRB.  
This subset typically includes only chemical shift assignments, but not spin 
systems and resonances.   As has already been described, these elements are 
key components of NMR analysis.  Third, the deductive process of reasoning 
is not captured.

Other significant efforts include SPINS \cite{baran2006spins}, 
Sesame \cite{sesame}, and 
the NESG’s efforts \cite{nesg2005nmr}.  However, much of the previous 
work in this area has focused on project management rather than reproducibility.
SPINS: claims to "organize and archive intermediate and final results", but 
this only refers to output files.  Sesame:  project management for 
high-throughput studies.  NESG’s platform describes a suite of software for 
automated NMR analysis, a bit of integration, and the need for occasional 
manual validation and editing.  No current systems provide functionality for 
the process of sequential deduction.

What is still missing is an approach and tooling for collecting all the primary 
data and meta data of NMR spectral analysis and structure determination.  Much 
time and effort is expended in these stages, but the data is not recorded.  The 
result is that the final data sets deposited into the BMRB are incomplete.

Irreproducible NMR spectral analysis and structure determination causes 
several problems.  As was mentioned in the previous section, the value and 
quality of irreproducible NMR analyses are difficult or impossible to judge; 
irreproducibility limits the ability to transfer knowledge and techniques 
(for interpretation of spectra, resonance assignment, stereospecific resonance 
assignments, etc.) effectively between scientists, as well as preventing close 
collaborations during data analysis and leading to time wastage as 
irreproducible results are discovered and following up on them is found to be 
impossible.  In addition, irreproducibility renders error detection and 
correcting difficult, because the data that would show when, why, and how an 
error occurred would be lost.  It also causes the teaching of analysis methods 
to students and other newcomers to be difficult due to implicit, missing data; 
by capturing and making explicit these data, a more complete picture of the 
process can be discussed and shown.  Finally, irreproducible data may be less 
amenable to future reinterpretation; reinterpreting data is necessary when 
augmenting a data set with additional results, which may fill in missing 
pieces, but may also show the original analysis to be in error.  In short, 
reproducibility of NMR spectral analysis and structure determination will lead 
to better quality results.


\section{An Approach for Reproducible Analysis}
This section outlines one possible approach for rendering analysis reproducible.
As was earlier stated, the key deficiencies causing irreproducibility are 
missing primary data (both extraneous results as well as intermediates), 
missing meta data relating to the deductive process of reasoning employed 
for manual analysis, and undeposited data (both primary and meta).  Therefore, 
the solution will focus on how to capture those data.  In addition, notes for 
identifying potential troublesome and confusing results will be used, as they 
provide a valuable service to future perusers as well as future 
reinterpretation by highlighting mistakes and ambiguities.

\begin{itemize}
  \item Problem: lost primary data:  extraneous results.  Standard approaches use the 
assumption that all peaks are true signal, with no provision for storing peaks 
determined to be processing artifacts or noise.  Such spurious peaks are simply 
deleted and do not show up in the final results.  This is a problem because the 
fact that a peak was found, and later interpreted as noise does not show up in 
the final data set.  The same problem applies to spin systems that are found 
but can not be assigned to any residue of the sample of interest, or are 
believed to correspond to atoms of a contaminant.  Such spin systems should 
be represented in the final data set.

Our approach is to allow any number of peaks and spin systems, and to 
augment them with additional data fields which distinguish between signal, 
noise, contaminants, etc.  This allows one to make a critical distinction 
between: 1) finding/recording a peak based purely on characteristics of 
the spectrum such as volume, height, relative height compared to noise, 
lineshape, and linewidth, and 2) interpreting a peak as signal, noise, 
etc. (and the same for spin systems).  Even peaks and spin systems for 
which no analysis is made can be kept in the data set without encumbering 
assignment of true peaks and spin systems.

  \item Problem: lost primary data: intermediate results.  The output of a 
computational tool, whether used for peak-picking, spin system construction, 
or sequence-specific spin system assignment, is often modified to correct 
mistakes.  This introduces a discrepancy between the output of the tool 
given the input and a suitable parameterization, and the final data set.  
By capturing a snapshot of the output of the tool immediately after it is 
run, and before modifications are made, this discrepancy is rectified.
Similarly, during manual analysis and modification of results, the state of 
the data is continually changing and determines which analyses may be made.  
For example, assignment of a spin system to a residue may allow a further 
unambiguous assignment of a different spin system to a residue (an assignment 
which previously would have been ambiguous) by eliminating one of two 
assignment possibilities based on matching amino acid type.  This shows that 
the state of the data determines what deductions can be made.  Therefore, it 
is important to capture these intermediate states before and after deductions.
The core of the strategy is based on that used by Version Control System (VCS) 
software tools, which are commonly applied for managing source code bases of 
software projects \cite{loeliger2012git}
[references to CVS, SVN, Git, Mercurial, Darcs].  

These tools were originally implemented in order to manage the change in 
source code over time, while retaining the ability to easily inspect past 
states of the code.  It was found that application of such tools led to 
large increases in productivity, robustness, correctness, and reduced 
faults [reference].  The core of a VCS is a model for change in data 
over time by storing multiple versions.  Versions are snapshotted, 
descended from previous snapshots, and annotated with a commit message 
which describes the what and the why of the change.  Similarly, in NMR 
during the process of sequential deduction, intermediate states form a 
chain.  By capturing these intermediates, similar advantages are gained 
(as in VCS).  While taking multiple snapshots of a large data set may 
seem wasteful of storage space, it is important to note that there are 
several approaches for compressing the snapshots to eliminate duplication; 
this essentially reduces the wasted space to zero.

  \item Problem: lost meta data: deductive reasons used.  This data describes the 
deductive process of reasoning employed in manually interpreting a feature 
of the primary data.  It is important because it provides the explanation 
of why something was done.

In a VCS, each snapshot is annotated with a commit message that describes 
the how and why of the snapshot -- what problems does it address, what 
features does it add.  Similarly, for capturing the data of the process of 
sequential deduction, the reasoning used describes the change to the data 
set and why it was made.  In order to support the capture of this meta data, 
an extensible library of commonly used deductive reasons enables the user to 
quickly and easily make deductions and describe the reasoning used.  These 
deductive reasons are then referenced in snapshot annotations.

  \item Problem: undeposited data (both primary and meta).  Current depositions to 
the BMRB do not include extraneous results, intermediates, or deductive meta 
data.  

A collaboration with the BMRB will extend the NMR-Star data dictionary to 
support this additional data.  Once the NMR-Star data dictionary supports 
it, spectroscopists will be able to submit complete data sets.

  \item Problem: lost meta data: notes.  As notes indicate the deficiencies and 
potential problems present in a data set, they are valuable to future 
perusers as they highlight how a data set is flawed and how it can be improved.
The CCPN and BMRB data models will be extended to support notes which may 
refer to any aspect of any other piece of primary data in the data set.
\end{itemize}

Additionally, this section will discuss how to effectively put this strategy 
into practice, covering common roadblocks and problems as well as tips and 
suggestions.


\section{Reproducible NMR Data Sets}
In order to prove the utility of the reproducibility approach and annotation 
model in practice, it was applied to a full-scale protein structure 
determination process.  Starting with time-domain data sets of the protein of 
interest, the structure determination process was carried out from start to 
finish, including peak-picking, sequence-specific assignment, NOESY analysis 
and structure calculation.  Intermediate snapshots were captured and 
appropriately annotated.  This data set may now be found deposited in the BMRB.

The data model used was based on CCPN’s data model \cite{ccpn}, with several 
extensions as previously noted to enable reproducibility.  A library of NMR 
phenomena and their use as deductive inferences rules was constructed.  
These rules were applied for snapshot annotation.

Whereas a single implementation of the previously described strategy is 
here described, in principle, the approach is platform-agnostic and 
therefore could be implemented and used by other research group, or added 
into an existing tool as an extension.


\section{Software for Practical Reproducibility}
Software tooling comprises a significant portion of enabling practical 
reproducibility.  High-quality tools can make reproducibility easy, pleasant, 
and safe (in the sense of not error-prone), without placing additional 
unreasonable time, effort, and education demands on potential users.  This 
section will explore a suite of software designed to enable and support 
reproducibility of NMR analysis in various ways.
	
First, a Sparky reproducibility extension has been developed.  Sparky 
\cite{sparky} is a popular tool for analysis of NMR data.  One of the major 
strengths of Sparky is its extensibility through user-defined Python modules.  
While its core is written in C++ and is not extensible, a Python module can 
be added without needing to touch the C++ core.  This extension .... % TODO 

Second, a library for reading and writing of NMR-Star files was implemented.  
NMR-Star is the standard format used by the BMRB for the deposition and 
archival of NMR data.  By storing ongoing data analysis in NMR-Star files, 
users gain the benefits of data integration with the BMRB -- analysis results 
can be uploaded and thereby shared with fellow researchers.  The approach 
taken in implementing and using this parser represents a radical departure 
from standard NMR software techniques; the approach ensures that the 
software will remain easily usable and maintainable as NMR data expands 
and matures.

Third, two tools for working with time- and frequency-domain data sets:  
Connjur Spectrum Translator (ST) and Connjur Workflow Builder (WB).  ST 
translates between various formats of time- and frequency-domain spectra; 
such a tool is necessary because of the input and output requirements of 
many spectral processing tools.  WB provides a high-level interface to 
spectral processing and stores the parameterization, functions, and 
intermediate data sets in a central, relational database.  This means 
that the stage is reproducible.

Fourth, a sample scheduler has been implemented.  This tool facilitates the 
creation of non-uniform sample schedules, which are used to collect time-domain 
data which are non-uniformly sampled in the indirect dimension(s).  Non-uniform 
sampling can help decrease the amount of data collection time required, and 
also help in avoiding the penalties imposed by the necessity of sampling past 
the Rovnyak limit \cite{rovnyak2004accelerated}.  This project included:
\begin{enumerate}
  \item a novel data model of sample schedules which included non-uniformity not 
only in the time dimensions but also in the number of transients per FID and 
the quadrature
  \item a collection of common algorithms, gathered from descriptions in the 
literature and implemented
  \item a reproducible and replicable approach to capturing the parameters of 
sample schedule creation
\end{enumerate}


\section{CONNJUR is Free and Open Source}
All software developed by the CONNJUR team is released under standard open 
source licenses and is freely available on our website.  The open source 
movement first became popular as a means for users to get control and legal 
rights over software that they had purchased.  This level of ownership is 
important because it enables users to freely and perpetually use, share, 
inspect, fix, maintain, and improve their software.  Such considerations 
become important in view of the rapid changes that scientific fields -- 
including NMR -- undergo: new datatypes, analyses, statistical measures, 
and protocols are developed, requiring updates to old software or entirely 
new software to be written from scratch.  Open source software provides 
additional value in the context of reproducibility: in order to replicate a
study, one must have access to the exact same computational tools that the 
original study used.  This involves both physical access -- in the sense of 
being able to get a program loaded onto a computer -- as well as licensing 
issues: does the second group have the rights to use the software in the 
exact same way as in the original study.

Our belief is that open source software can help mitigate these and other 
problems, as well as aid the field in more effectively dealing with its 
nascent software problem, by leading to adoption of a community development 
model and increased sharing, reducing the barriers to future progress in 
the field.  We can only hope that other research groups place as much value 
as we do on open source licenses, and that adoption of open source development 
models will increase.  Too much scientific software has been placed under lock 
and key and we have suffered as a result: suffered from inadequate software, 
arrested progress, unfixable bugs, abandonware.


\section{Scope and Significance}
Reproducibility is a key enabler of the success of the scientific approach to 
acquiring knowledge.  This work inspects reproducibility in the field of NMR, 
defines the requirements for data analysis that must be met in order to achieve 
reproducibility, and identifies where current practices fall short.  To remedy 
this situation, a strategy for reproducible data analysis is presented.  This 
strategy is made practical by means of a formal model, support from the BMRB, 
and a software implementation.





\chapter{NMR Data Analysis is Irreproducible}

% not sure what the difference is between `unsrt` and `ieeetr`
% also see `natbib` for another possible alternative
%\bibliographystyle{unsrt}
\bibliographystyle{ieeetr}
\bibliography{thesis}

\end{document}

