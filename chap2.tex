\section{Spectral processing}

The raw data collected from an NMR spectrometer is referred to as 
time-domain data.  In a typical NMR experiment, these data represent the 
sum of multiple decaying sinusoids.  These FIDs are converted to 
frequency-domain spectra which are used in further analysis.  The goal of 
this phase is to construct a frequency spectrum which indicates the resonance 
frequencies of the atoms that were observed in the experiment.  A common tool 
for such a transformation is the Fourier Transform, which is able to convert 
a data set consisting of multiple overlapped signals to a frequency spectrum.  
Due to relaxation decreasing the amplitude of an NMR signal over time, peaks 
have an intrinsic linewidth in the frequency spectrum.

Considerations include:  minimization of processing artifacts; signal-to-noise 
ratio; accounting for water lines; avoiding rolling baselines and baseline 
offsets; linewidth; phasing; lineshape; sync wiggles.  Multiple software 
packages exist for carrying out this conversion \cite{nmrpipe, rnmrtk}.
These packages include functions for processing the data in specific ways to 
guarantee desirable qualities.  A typical procedure for spectral processing 
involves the sequential application of multiple functions from one of these 
packages.  At each stage, the input is a data set and associated metadata, 
which includes information such as spectral width, dwell time, and number of 
points.  Each function may require the setting of one or more parameters in 
order to proceed.  Thus, in addition to the final frequency-domain spectrum, 
the process to create generates several intermediate data sets, several 
intermediate metadata sets, and the sequence of functions used along with 
their parameterizations.  Previous work from our lab has enabled the 
convenient collection of necessary metadata during spectral 
processing \cite{connjur-wb}.


\section{Peak picking}

The Fourier Transform of a decaying sinusoid produces a frequency-domain 
spectrum with a peak at a frequency matching the oscillation frequency of 
the sinusoid.  Since NMR time-domain data consists of multiple decaying 
sinusoids caused by atoms resonating at characteristic chemical shifts, 
the frequency domain spectrum will contain peaks for every oscillating 
sinusoid present in the time-domain data.  Identification of the positions 
of these peaks and of their attributes such as shape, intensity, and width 
is the next important step after spectral processing.  The goal of peak 
picking is to build collections of peaks.  Thus, if peak picking proceeded 
perfectly, the final peak lists would only contain true signal peaks, and 
would contain all true signal peaks.  
[example] [example with noise] [example with artifacts] [example with small signal peaks]  
In addition, for each peak, all of the following should be true: 
\begin{itemize}
  \item the correct frequency position for each dimension are determined
  \item the correct lineshape and linewidth are obtained
  \item the correct intensity is measured
\end{itemize}

Correct results are important because they form the basis for the construction 
of GSSs, the assignment of chemical shifts to atoms, and the interpretation of 
NOESY spectra which give rise to distance restraints as a preliminary to 
structure calculation.  Incorrect peak identification or position can result 
in misinterpretation of NOESY spectra, which could lead to false distance 
restraints between atoms which are in fact very far apart in the actual 
protein structure.
	
However, correct and complete peak picking is complicated by false peaks, 
such as random noise and artifacts of the processing and/or sampling schemes, 
as well as peaks caused by contaminant molecules or the solution itself if it 
contains NMR-active nuclei.  peak picking is further complicated by overlap -- 
when two or more peaks are closer than the spectral resolution, resulting in 
distortion of lineshape, position, and intensity measurements; severe cases 
of overlap can even cause multiple peaks to appear as a single one.  Another 
difficulty of peak picking is low-intensity peaks, which causes them to be 
nearly indistinguishable from the noise and baseline; this may be due to 
sample instability such as aggregation or precipitation, or low sample 
concentration
\cite{picky, munin, korzhnev2001munin, apart,
autopsy, pine}
\cite{williamson2009automated, guntert2009automated, altieri2004automation,
baran2004automated}.
In such cases, a naive peak picker based on intensity will necessarily 
generate either false positives or false negatives.

Estimates of the amount of false positive and false negative peaks picked 
range from very low (10-40\%) to very high (70-135\%) \cite{pine}. 
The quality of the results generally depends on characteristics of the 
spectrum, especially the signal/noise ratio, and resolution, as well as 
characteristics of the molecule including T2 (which has an effect on peak 
width) and number of atoms -- the issue being that more atoms means more 
peaks, which means a higher chance of overlap.

In order to address these problems, various approaches have been tried.  
Programs have been implemented which analyze peak shape and intensity as 
well as local noise level \cite{munin, autopsy}.  Other approaches 
apply more advanced mathematical models to identifying peaks correctly.  
Deposited data available in public archives may also be used 
These approaches have been implemented as computational tools 
\cite{williamson2009automated, guerry2011automated}.
However, none of these approaches yields perfect 
results \cite{guerry2011automated}.  Manual intervention during peak picking 
is still preferred, at least to a certain extent, and most peak picking 
programs allow and encourage semi-automated interaction in order to clear 
up troublesome features.  Manual intervention is useful for identifying 
overlap, noise, artifactual, and extraneous peaks, 
and plays an important role in ensuring that the quality of the peak picking 
is sufficiently high to enable further analysis \cite{guntert2009automated}.  
Manual intervention is often accomplished based on knowledge outside of the 
spectrum: existence, position, and shape of peaks in other spectra, knowledge 
of the solvent, characteristic artifactual patterns caused by a specific 
processing scheme, knowledge of the local dynamics of a small region of the 
protein.  \cite{williamson2009automated, guntert2009automated, 
altieri2004automation, baran2004automated}
Because of this, peak picking can often not be completely finished until 
later analysis has been accomplished.


\section{Spin system and resonance construction}

After collecting and peak-picking a specific set of spectra designed to 
capture the chemical shifts of the backbone H, N, C, CA as well as the CB, 
the peaks are grouped into what is known as pseudoresidues \cite{mars} 
or generic spin systems (GSS) 
\cite{saga, ezassign, pistachio, autoassign1997, autoassign2001}.  
A GSS is the NMR-visible incarnation of covalently bonded, magnetically 
interacting nuclei through j-coupling.  GSSs are defined as having a root 
resonance or resonances -- typically an amide H-N group due to the high 
dispersion in resonance frequency of such groups which enables easier 
unambiguous identification of each H-N group, and the relatively high 
sensitivity of H-N groups in NMR experiments -- and additional 
covalently-bonded resonances.  The precise extent of a GSS is in principle 
determined by the available NMR experiments which exploit the network of 
covalently bonded, coupled nuclei.  In practice, a GSS often is initially 
comprised of a backbone H-N, CO, CA, CB, CO(i-1), CA(i-1), and CB(i-1).  
At a later stage, GSSs are often augmented with additional sidechain resonances.  
In addition to GSSs of backbone resonances, H-N-rooted GSSs typically are 
visible for Asparagine and Glutamine sidechains, and smaller GSSs of 
Tryptophan sidechains are visible.  Arginine sidechains may also give rise 
to an H-N-rooted GSS under certain experimental conditions. 
	
Typical NMR experiments for observing CO, CA, and CB nuclei also include H-N.  
This is the practical reason for treating GSSs as H-N-rooted: across multiple 
spectra, the same H-N group will appear at approximately the same frequency.  
Based on the consistency of of H-N chemical shifts, the peaks can be grouped; 
given an NHSQC, HNCO, HN(CA)CO, HNCACB, and HN(CO)CACB spectrum, each backbone 
H-N-rooted GSS would be expected to have:
\begin{itemize}
 \item H: 10 measurements, NHSQCx1, HNCOx1, HN(CA)COx2, HNCACBx4, HN(CO)CACBx2
 \item N: 10 measurements, NHSQCx1, HNCOx1, HN(CA)COx2, HNCACBx4, HN(CO)CACBx2
 \item CO: 1 measurement, HN(CA)COx1
 \item CO(i-1): 2 measurements, HNCOx1, HN(CA)COx1
 \item CA: 1 measurement, HNCACBx1
 \item CA(i-1): 2 measurements, HNCACBx1, HN(CO)CACBx1
 \item CB: 1 measurement, HNCACBx1
 \item CB(i-1): 2 measurements, HNCACBx1, HN(CO)CACBx1
\end{itemize}

The difficulty in constructing these GSSs correctly and unambiguously stems 
from the inherent nature of NMR data.  First, the success of the standard 
suite of experiments rooted in H-N -- NHSQC, HNCO, HNCACB, etc. -- depends 
on \cite{autoassign1997}: % page 600 -- `reliability`
\begin{enumerate}
  \item good dispersion, i.e. no overlap, otherwise it is hard to tell which 
  peaks belong with which H-N-rooted spin system.
  \item the H-N chemical shifts being nearly identical across all spectra.  
  This may not be the case if there are variations in the sample or the 
  temperature.  The Block-Siegert shift and experimental error also can have 
  an effect on chemical shift.
  \item atoms appearing at a single chemical shift.  If there are multiple 
  conformations or chemical heterogeneity \cite{autoassign1997}, 
  an atom may appear at multiple chemical shifts and appear to be two 
  different resonances.
  \item the presence of an H-N group -- Proline is a notable exception, and 
  so it does not show up in experiments which rely on the presence of an H-N group
  \item extraneous peaks which do not seem to fit into a spin system, or 
  peaks which do not seem to match peaks in other spectra
  \item accurate (or at least consistent) spectral referencing.  
  Misreferenced spectra will cause the same atom to show up at different 
  chemical shifts across multiple spectra.
  \item quality of peak-picking \cite{autoassign1997}: 
  chemical shifts, false positives, false negatives, extraneous peaks
\end{enumerate}

Relatively complete and correct peak-picking is a prerequisite for GSS 
construction \cite{mars}.  Computational approaches 
for GSS construction tend to require manual assistance in some cases 
\cite{autoassign1997, mars}.  Incorrect or 
incomplete GSSs will have negative effects on the quality of later 
analysis, and so it is important to obtain correct GSSs; many such 
tools assume that manual intervention will verify and, if necessary, 
correct the GSSs; this allows the tools to be conservative in their 
predictions \cite{autoassign1997}.  However, it may not be 
possible to unambiguously and completely construct GSSs until the results 
of later analysis are available: some approaches use NOESY peaks and 
assignments as well as structure results to verify and correct GSSs 
\cite{autoassign1997}.


\section{Backbone spin systems and resonances: typing and sequence-specific assignment}

Once GSSs have been constructed, analysis continues with four simultaneous subtasks:
\begin{enumerate}
  \item assignment of resonance to atomtype
  \item assignment of amino acid type to GSS
  \item assignment of GSS to GSS sequentially
  \item assignment of GSS to specific residue of sample of interest
\end{enumerate}

\subsection{Assignment of resonance to atomtype}
GSSs are composed of resonances \cite{ccpn}.  Resonances are 
similar to GSSs in their relationships to atoms and residues: resonances can 
be identified consistently across spectra without reference to a specific 
atom; later interpretation is expected to assign the resonance to an atom.  
The resonance datatype captures this concept, an intermediate between 
frequency spectra and atoms.  Before assigning a resonance to a specific 
atom, the atomtype of a resonance may be assigned.  In an HNCO experiment, 
this is typically straightforward, because for each backbone spin system, 
the H dimension always corresponds to the backbone H, the N dimension always 
corresponds to the backbone N, and the C dimension always corresponds to the 
backbone C(i-1).  However, the situation is more complicated in an HNCACB 
experiment, as there are generally four choices of atomtype assignment for 
the C dimension:  CA, CB, CA(i-1), and CB(i-1).  Thus, the resonance given 
by the C dimension of each peak must be assigned one of these choices.  
Reasons for choosing a specific assignment include peak sign, as well as 
chemical shift compared to statistics available in the BMRB.  In addition, 
the overlap between experiment pairs such as the HNCACB and CBCA(CO)NH 
facilitates resonance-atomtype assignment: while the CA(i-1) and CB(i-1) 
are expected to appear in both experiments at the same chemical shift for 
a given backbone H-N root, the CA and CB are expected to appear only in the 
HNCACB spectrum.

\subsection{Assignment of amino acid type to GSS}

Correspondingly, GSSs are also assigned amino acid types.  This phase 
interacts strongly with the assignment of atomtypes to resonances, in 
that the possible atomtypes to which a resonance may be assigned depends 
on the amino acid type, and the expected chemical shift ranges for various 
atomtypes depends on amino acid type as well.  For instance, GSSs assigned 
to the Glycine amino acid type should not have a CB; and the CB resonance's 
chemical shift of a GSS assigned to Alanine is expected to be very different 
from all other CB chemical shifts.  Backbone amino acid types may be split 
into several categories \cite{saga} based on BMRB statistics for 
CA and CB chemical shifts \cite{bmrb}:
\begin{enumerate}
  \item Ala
  \item Gly 
  \item Pro
  \item Ser, Thr
  \item Val, Met, Lys, His, Arg, Glu, Gln, Trp, Cys
  \item Asp, Asn, Ile, Leu, Phe, Tyr
\end{enumerate}
However, GSS typings are complicated by several factors.  First, GSS typing 
requires correct and complete GSS construction.  Second, correctly assembled 
GSSs may include overlapped or extraneous peaks, expected peaks (based on a 
spectrum's typical results) may also be missing.  Third, most GSSs can not 
be uniquely typed based solely on CA and CB chemical shifts, as groups 5 and 
6 (above) as well as 4 are ambiguous.  Fourth, sidechain GSSs must be 
identified and separated.

\subsection{Assignment of GSS to GSS sequentially}

Sequential GSS assignments are made using an important feature of common NMR 
experiments such as the HNCACB and HN(CA)CO: natural overlap between 
sequential GSSs.  This phenomenon is enabled by the similar coupling 
constants between the backbone N and the CA and CA(i-1) nuclei, which allows 
a single experiment to simultaneously correlate the H-N with both nearby CAs, 
and through CAs covalent bonds to additional atoms.  This phenomenon is 
exploited such that, in principle, each CA, CB, and C atom is observable 
through standard NMR experiments from two backbone H-Ns;  sequential GSSs 
are expected to have CA/CA(i-1), CB/CB(i-1), and CO/CO(i-1) resonances at 
identical chemical shifts.  This duplication enables sequential assignment 
of GSSs.  Note that there is substantial interaction between atomtype-resonance 
assignment and sequential GSS assignment: assigning two GSSs sequentially 
implies the CB vs CB(i-1), CA vs CA(i-1), and C vs C(i-1) atomtype assignments 
of the resonances in both GSSs; knowing the atomtype-resonance assignments of 
two spin systems can prevent their sequential assignment (if, for example, the 
matching resonances are both CB(i-1)); and not knowing the atomtype-resonance 
assignment implies that the sequential GSS assignment may be invalid.  
Sequential GSS assignment is complicated by: 
\begin{enumerate}
  \item missing peaks, possibly caused 
  by local dynamics, which reduce the number of overlapping resonances between 
  potential sequential GSSs, and can also disrupt atomtype assignments of 
  resonances; 
  \item extraneous peaks, which may be false positives or caused by 
  multiple conformations of the protein, causing incorrect matches
  \item degeneracy of chemical shifts:  given two GSSs with identical CA(i-1) 
  and CB(i-1) resonances, as well as a third GSS with matching CA and CB 
  resonances, it is impossible to unambiguously assign sequentially solely 
  on the basis of chemical shift matching between the two GSSs 
  \cite{autoassign1997}.
\end{enumerate}

\subsection{Assignment of GSS to specific residue of sample of interest}

The final piece of backbone analysis is assignment of GSSs to specific 
residues.  Since backbone GSSs are H-N-rooted, a GSS is assigned to a 
backbone-amide; this implies the assignment of resonances to atoms as well, 
based on matching of atomtypes.  When a typical GSS is assigned to a residue, 
the H, N, C, C(i-1), CA, CA(i-1), CB, and CB(i-1) atoms will be assigned 
resonances as well.  Sequence-specific assignment interacts heavily with 
sequential GSS assignment, because the protein sequence must be compatible 
with the GSS sequence, where `compatible' means that the amino acid types 
of the GSS match those of the protein sequence.  Note that full assignment 
of amino acid type to GSS is not a prerequisite for GSS-residue assignment; 
in fact, GSS-residue assignment may lead to GSS-amino acid type assignment 
for sequentially connected GSSs.  GSS-residue assignment is facilitated by 
long chains of sequential GSSs in which some of the GSSs are typed as Serine, 
Threonine, Glycine, or Alanine.  The longer a GSS chain, the fewer places it 
might possibly fit into the protein sequence \cite{saga}.  Also, 
as sequence-specific assignment proceeds, the number of unassigned GSSs and 
residues decreases; the result is that initially ambiguous assignments become 
unambiguous as choices are removed.  Conversely, complications arise from 
incomplete sequential GSS assignments resulting in short, ambiguous chains.  
The presence of prolines generally ends chains due to the lack of a backbone 
H-N group.  Missing GSSs also terminate chains.  Relatively few Ser, Thr, Gly, 
and Ala residues means the number of unambiguous anchor points will be lower.


\section{Sidechain: spin system and resonance assignment}

The next group of experiments collects chemical shifts of sidechain atoms.  
These experiments include the HBHA(CO)NH, C(CO)NH-Tocsy, HC(CO)NH-Tocsy, 
HCCH-Tocsy, and aromatic Tocsys.  The purpose of these experiments is to 
obtain the chemical shift values of sidechain resonances of protons, since 
proton frequencies are necessary in order to interpret NOESY spectra.  To 
interpret these spectra, the peaks must be assigned to GSSs and atomtypes 
must be assigned to the new resonances. While several of these experiments 
are also rooted in backbone H-N groups, facilitating the addition of peaks 
to the correct GSS, others -- such as the HCCH-Tocsy -- are not.  These are 
analyzed by the matching of resonance chemical shifts with those from other 
experiments targeting sidechains.  Atomtype-resonance assignments can generally 
be made with reference to compiled BMRB statistics.  Complications in this 
phase include: stereospecificity -- atoms such as HA2 and HA3 may give rise 
to different chemical shifts, but resolving the correspondence may be 
impossible without further data; overlap -- especially in the HCCH-Tocsy 
where sidechains of the same amino acid type but different residue may have 
many closing matching chemical shifts; overlap between resonances within the 
same GSS, especially in Leu and Ile; missing and extraneous data; and the 
difficulty of both obtaining and unambiguously interpreting aromatic data.  
New approaches for sidechain data collection and assignment have recently 
been developed \cite{mobli2010non, hiller2008apsy} which seek to address 
these issues by reducing ambiguity of chemical shifts.


\section{NOESY peak-picking and assignment}

In NOESY spectra, each true peak indicates a pair of protons within 
approximately 5 Angstroms of each other.  This is different from the 
correlation spectra discussed earlier, in which peaks indicate covalently 
bonded atoms; NOESY spectra do not depend on covalent bonds but rather 
depend on spatial proximity.  Thus, each NOESY peak contains some information 
about the actual three-dimensional structure of a molecule, if it can be 
determined which protons give rise to the peak.  Analysis of NOESY spectra 
requires chemical shift assignments of atoms, and uses that information to 
determine the protons involved in a peak.  Considerations used to analyze 
NOESY spectra include: symmetry -- a peak is expected to correspond to a 
matching peak with the frequencies of the two 1H dimensions swapped; patterns 
based on known proximity of atoms from the primary sequence giving rise to 
many short-distance NOE peaks; network anchoring.  Complications include: 
\begin{enumerate}
  \item overlap caused by degenerate chemical shifts of protons, leading to 
ambiguous interpretations of peak assignments; this can be greatly mitigated 
by the use of an extra dimension:  15N- or 13C-edited NOESY spectra reduce 
the ambiguity
  \item incorrect or incomplete chemical shift assignments
\end{enumerate}

NOESY assignment is generally done automatically \cite{cyana2004, aria2003}.  
NOESY peak-picking may be automated as well \cite{munin, korzhnev2001munin}.


\section{Structure calculation}

Cyana is able to calculate a three-dimensional structure from NOESY peaks, 
chemical shift assignments, and distance restraints \cite{cyana2004, aria2003} 
using an iterative approach to NOESY peak assignment and building structural 
models.  It also requires secondary structure information as input, which can 
be calculated from chemical shift assignments using a program such as 
\cite{talos+}.  Chemical shift assignments may also be used to 
calculate potential structures \cite{cs-rosetta}.  Additional programs 
may be used to refine structures \cite{amber, xplor-nih}.  This phase 
generally does not require manual intervention.


\section{Discussion}

The inherent NMR issues of ambiguous, missing, and extraneous data cause 
problems throughout the entire analysis process.  Correctly dealing with 
these issues is difficult, but absolutely critical in order to obtain 
high-quality results \cite{williamson2009automated, guntert2009automated, 
altieri2004automation, baran2004automated}.  As yet, computational tools 
are not able to deal perfectly with these issues.  
Most tools have several basic limitations: 
\begin{enumerate}
  \item they require high-quality input in order to function correctly 
  \cite{saga, abacus_assignment, mars, autoassign2001, ezassign, pine, cyana2004}; 
  this input is generally assumed to have been manually prepared in order 
  to meet the stringent quality requirements of completeness and absence of 
  extraneous results
  \item even with high-quality input data, tools are not able to produce 
  perfect results 
  \item tools perform differently in different contexts, although 
  performance generally decreases as protein size increases and spectral quality 
  decreases
  \item manual verification and correction of the results is assumed, 
  even for tools that claim to be fully automated 
  \cite{williamson2009automated, guntert2009automated, altieri2004automation,
  baran2004automated}
\end{enumerate}

A key limitation of many analysis tools is that they are viewed as part of a 
pipeline: input is transformed into output, which becomes the input for the 
next stage, and so on \cite{pine}.  This may explain why these tools 
are not able to produce better results:  full analysis, including peak-picking, 
GSS construction, assignment, and so on, can make use of information from 
multiple stages.  Humans, performing manual interventions, are able to bring 
this additional information to bear to make specific interpretations; however, 
many of these tools do not.  Thus, PINE and related efforts  
are an exciting effort to loosen these artificial restrictions.  Initial 
results are promising, and show a marked improvement, although manual 
intervention is still assumed to be necessary in order to obtain the best 
results \cite{pine}.  Further tools such as \cite{shiftx2, cheshire}, 
which calculate chemical shifts from structure, provide additional means 
for bringing additional information to bear, helping to validate assignments.

Another exciting development is the rise of probabilistic methods 
\cite{saga, pine}.  These methods reflect the reality that the 
confidence of a specific interpretation depends on the exact state of the 
data; in other words, an assignment which is 50\% confident given only an 
HNCA spectrum may become 90\% confident if an HN(CO)CA spectrum is added.  
The significance of this confidence level is that it enables easy tracking 
of ambiguous and/or low-confidence interpretations -- i.e. those that stand 
to benefit from collecting additional data sets.  By including confidence 
values on all assignments, an understanding of the troublesome areas is 
facilitated.  This helps to reduce the cost of cascading errors -- if the 
uncertainty is tracked as a confidence level, further interpretations based 
on a highly uncertain datum will also receive low confidence levels.  In 
addition, confidence levels are an alternative to the inherent balance 
between completeness and correctness -- it is no longer necessary to 
sacrifice one for the other \cite{autoassign2001, pine}.


\section{Conclusions}

Manual analysis plays a critical role in NMR data analysis, due to the inherent 
issues involved in analysis and to the large amount of information which must 
be brought to bear to solve difficult cases.  Although spectral processing, 
NOESY assignment, and structure calculation have been automated, peak-picking, 
GSS construction, and assignment of both resonances and GSSs have not been 
completely automated.  Manual intervention is assumed to be necessary by 
most tools, even automated ones, to ensure the completeness and correctness 
of results.  However, despite the importance that manual intervention plays 
in analysis, the specific modifications made and their reasons for -- 
which may be quite complicated -- are not captured \cite{guntert2009automated}.  
Thus, the metadata of manual intervention is lost, and analysis is 
irreproducible.

