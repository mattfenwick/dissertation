\chapter{NMR Spectroscopy of Proteins}
\begin{center}
  \textit{Nature's imagination far surpasses our own.}

 - Richard Feynman
\end{center}



\section{Protein NMR}

\subsection*{Sample preparation}

The first step of an NMR analysis process is to obtain a sample of the protein
of interest.  NMR is a relatively insensitive technique, and so it is important
to get a sample with relatively high concentration compared to other structural
biology techniques, often in the millimolar range.  
One technique for obtaining such a high concentration of sample protein is
to express the protein in bacteria.  Bacteria are relatively cheap and easy
to grow, and their food sources can be regulated to provide NMR-friendly 
isotopes such as \nmrisoc{} or \nmrison{} if desired, by controlling the growth media.
The protein is produced by transforming bacteria through adding a plasmid,
and then inducing transcription of the plasmid followed by translation.  
Finally, the protein is isolated and purified \cite{derome1987modern}.

\subsection*{Data collection}

Assuming the purified, high-concentration protein sample does not aggregate
or denature, NMR experiments are run by inserting a tube of the sample into 
a spectrometer, running sequences of radio-frequency pulses which selectively
excite nuclei within the sample, and observing the results.  The data are a 
superposition of decaying sinusoids of different frequencies and amplitudes;
the frequencies correspond to the precession at the Larmor frequency
of the various nuclei \cite{bloch1946nuclear}.

Various pulse sequences are used to probe specific functional groups within
the protein.  There are two major categories of pulse sequences, based on the
nature of the interactions they exploit: the first group exploits scalar 
couplings which exist between covalently-bonded nuclei, and are thus called 
"through-bond" experiments \cite{davis1985assignment}.  
The second group exploits cross-relaxation between proton pairs that are
dipole-dipole coupled; these protons are spatially close but do not have to 
be covalently bonded, and are thus called "through-space" experiments
\cite{solomon1955relaxation}.  The data produced by these two experiments are 
used differently.

Two important parameters of data collection are the dwell time and the number
of points collected.  The dwell time determines the range of frequencies that
can be distinguished.  The number of points collected is related to the digital 
resolution (the ability to distinguish between nearby, but distinct, frequencies).  
The acquisition time, which is the product of the dwell time and the number 
of points, is related to the resolution.

Maximizing sensitivity is helpful for later data analysis.
Sensitivity depends on the gyromagnetic ratios of nuclei and 
the strength of the external magnetic field due to the magnitude of the
difference between the high- and low-energy states of a spin-1/2 nucleus
according to the Boltzmann distribution.
Running an experiment multiple times and summing the results is a 
means of increasing sensitivity as the signal increases faster
than the noise, assuming random distribution of the noise
\cite{ardenkjaer2003increase}.

Referencing ensures that results are comparable from multiple
spectrometers.  While the absolute Larmor frequencies of nuclei vary depending
on the magnetic field of the spectrometer, the normalized values, when compared
to a known material are consistent \cite{wishart19951h}.  
These are known as chemical shift values, and are reported in parts per 
million of deviation.  A small amount of a known material is placed in the 
sample to provide referencing.

\subsection*{Spectral processing}

The time-domain experimental data are converted to frequency-domain spectral
data.  A decaying sinusoid in the time-domain becomes
a peak with a finite and non-zero linewidth at the corresponding frequency
in the frequency-domain.  The Fourier Transform \cite{cooley1965algorithm}
is a standard method for 
converting between time- and frequency-domains, and is often used on NMR data.
Through appropriate use of scaling and normalization, the frequency axis is
converted to chemical shifts.

The Nyquist theorem \cite{nyquist1928certain, shannon1949communication}
places bounds on the dwell time with respect to the 
final spectral width.  A poor choice of dwell time can lead to spectral 
aliasing, in which peaks appear in unexpected spectral regions because their
frequencies are outside the range supported by the chosen dwell time.
Two factors confound resolution:  coincidental
closeness of resonances frequencies, and experimental quality.  In general, 
larger proteins, which have more atoms than smaller proteins and therefore
more nuclei as well, have more resonance
frequencies in close proximity to each other, increasing the probability of 
overlap.  To increase resolution, data points are collected to 
longer times.  Care must be taken to avoid decreasing the sensitivity;
non-uniform sampling and Maximum Entropy reconstruction offer one means of so
doing \cite{rovnyak2004accelerated, hoch1985maximum}.

A peak in a through-bond spectrum and in a through-space spectrum have 
different meanings.  In a through-bond spectrum, a peak indicates the 
observation of several resonating nuclei connected by a small number of
covalent bonds.  However, a peak in a through-space NOESY spectrum indicates
the presence of two protons within approximately 5 Angstroms of each other
\cite{neuhaus1989nuclear}.

\subsection*{Peak picking}

Peak picking is the process of identifying signals in an NMR spectrum using
peaks as a proxy.  The position of the multi-dimensional peak indicates the
chemical shifts of the nuclei giving rise to it, and the amplitude may have
significance in some but not all experiments.

Peak picking would be if easy if several conditions were met by the spectra:

\begin{itemize}
  \item all expected signals appear
  \item all signals are easily distinguishable from noise and artifacts
  \item all signals are well dispersed from all others
  \item no unexpected signals appear
\end{itemize}
 
However, in practice, none of these conditions are met \cite{williamson2009automated}.  
Thus, expected signals are missing, some signals are close to the noise
level, some noise appears to be signal, some artifacts appear, signals overlap
to greater or lesser extents, and some unexpected signals appear, perhaps due
to contamination or multiple conformations \cite{baran2004automated}.

Therefore, accurate peak picking must deal with these problems, in order to
identify all the true signals, none of the false signals, and to correctly 
characterize the position and volume of the true signals.  Initial peak picking
is often performed using a computational tool, but there typically is some
level of manual intervention in order to correct mistakes and other problems
\cite{guerry2011automated}.

\subsection*{Chemical shift assignment}

Nuclei in NMR experiments resonate at characteristic frequencies; 
chemical shift assignment is the process of drawing the correspondence between
the resonance frequencies identified from picked peaks and the nuclei in the
protein of interest.  Assignment is typically accomplished using a set of
through-bond spectra, many of which are based around H-N groups and nearby
nuclei, and others which obtain the chemical shifts of the aliphatic sidechains
(both carbons and protons) and still others for the aromatic portions of
sidechains \cite{hnco, hncacb, cbcaconh, hbhaconh, cconhtocsy}.

Assignment proceeds through two key intermediate data types: generalized spin
systems (GSSs) and resonances \cite{bmrb, ccpn}.  
Resonances are the NMR incarnations of nuclei: a nucleus resonates at a 
characteristic frequencies across spectra.  GSSs
are similar to NMR incarnations of amino acid residues, but may span
multiple residues and are therefore networks of covalently-bonded resonances
\cite{abacus_assignment}.

Peaks are assembled into GSSs by exploiting the redundancy between several 
experiments: resonances appear in multiple spectra, at the same characteristic
frequency giving rise to peaks (signals), and this is used to match these
peaks into the same GSSs and resonances.  Peaks can also be matched within
a single spectrum into the same GSS, depending on the nature of the experiment
\cite{ccpn}.

GSSs and resonances are assigned an amino acid type or a nucleus type, 
respectively.  As can be found in the BMRB, there is large variation in 
average chemical shifts depending on amino acid type and nucleus type,
especially for serine/threonine, glycine, and alanine residues, for which the
CB, CA, and CB nucleis' chemical shifts are essentially unique
\cite{guntert2009automated}.

A second type of overlap is used in chemical shift assignment: due to the 
similar J-couplings of the CA both of the previous and same residue to the
backbone nitrogen, it is possible to correlate a backbone amide group with
both adjacent sidechains.  While the J-coupling to the CA is typically larger
than the J-coupling to the CA(i-1), the relative ratio is usually within a 
factor of two.  The practical implication is that the correlations to both
CA nuclei can be collected in the same experiment.
Thus, each sidechain may be correlated with two
backbone amide groups, causing their resonances to appear in conjunction with
two other groups.  As the nuclei resonate at a characteristic frequency, this
can be used to identify a sequential connection between the two GSSs based on
backbone amide groups.  Once a sufficiently long chain is built using these
sequential connections and the (possibly incomplete) GSS typings, the chains 
may be assigned to specific residues in the protein sequence.  Combined with
resonance typing, chemical shift assignments may be obtained
\cite{hnco, hncacb, cbcaconh}.

However, chemical shift assignment is complicated by missing, overlapped, and
extraneous signals, as well as ambiguities in GSS typings, resonance typings,
sequential GSS connectivity, and sequence-specific GSS-residue assignments
\cite{williamson2009automated, guerry2011automated}.
The ambiguities in typings are caused by the non-uniqueness of average chemical
shifts for most residue types (apart from glycine, alanine, serine, and 
threonine) \cite{bmrb}.  The ambiguities in sequential
assignments are caused by degenerate chemical shifts across multiple residues,
as well as by missing and extraneous signals, and those in sequence-specific
assignments are caused by non-uniqueness of the match between GSS typing
of a sequential chain and the primary sequence of the protein.

Accurate and complete chemical shift assignment requires nearly complete and
correct peak picking, as well as the presence in the spectra of nearly complete
expected signals, well-dispersed such that there is little to no overlap
\cite{guntert2009automated, guerry2011automated}.
It is often helpful to use a computational tool to quickly assign most of the
chemical shifts, but later to make manual interventions to fix mistakes and
assign any missed resonances \cite{baran2004automated}.

\subsection*{NOESY assignment}

NOESY spectra provide distance restraints between proton pairs.  In order to
make use of their latent structural information, the peaks must be assigned
to resonances and thereby to nuclei.  This is done with the help of the 
chemical shift assignments: NOESY peak cross-sections are matched to nuclei
based on similarity of the cross-section's chemical shift to that of the
resonance assigned to the nuclei.

However, NOESY data are heavily ambiguous, because there are typically several
resonances with chemical shift values close enough to match a single NOESY
peak cross-section.  There are several strategies for mitigating this problem.
One is to collect 3D or 4D NOESY experiments, in which the additional dimensions
correlate covalently-bonded \nmrisoc{} or \nmrison{} nuclei to the protons 
involved in the NOE interaction \cite{majumdar1993improved}.  
This approach greatly reduces the ambiguity.  Furthermore,
characteristic peak patterns are expected, such as intra-residue NOESYs between
protons of that residue, as well as NOESY peaks between protons of sequential
residues.  Another approach is selective labeling \cite{takeda2007automated}.

Accurate and complete NOESY interpretation requires nearly complete chemical
shift assignment.  Furthermore, some manual intervention in NOESY assignment
may be necessary to correct and validate troublesome cases, or to prevent
automated assignment programs from making mistakes 
\cite{guntert2009automated, guerry2011automated}.

NOESY data may be assigned manually, but are often assigned computationally
as well, or with a combination of the approaches.  The CYANA and ARIA structure 
calculation programs include facilities for automated NOESY assignment
\cite{cyana2004, aria2003}.  
A third tactic for dealing with ambiguous NOESY data is to iteratively reduce the
ambiguity using network-anchoring approaches, that use initial structure 
estimates to drive further unambiguous NOESY assignment in a self-consistent
cycle \cite{cyana2004, aria2003}.

\subsection*{Structure calculation}

There are several other types of structural information obtained through NMR
besides the proton-proton distance restraints provided by NOESY spectra.
Using a program such as TALOS+ \cite{talos+}, backbone torsion angles can be 
predicted.  TALOS+
uses the chemical shift assignments of backbone nuclei in conjunction with a
database search to make its predictions.  3-J-coupling constants can be 
related to dihedral torsion angles through the Karplus equation
\cite{karplus1959contact, karplus1963vicinal}.  Residual dipolar couplings
(RDCs) provide information about internuclear orientations.

These data are synthesized into a structural model by programs including CYANA,
ARIA, and XPLOR-NIH.  CYANA is useful for quickly obtaining coarse structure
estimates.  XPLOR-NIH is able to provide more detailed structural models, but
may take far longer to calculate a structure \cite{xplor-nih, cyana2004}.

\subsection*{Deposition}

The BMRB is the main repository for information derived using NMR spectroscopy.
A BMRB deposition may be prepared that includes chemical shift assignments,
peaks, peak assignments, binary spectral and time-domain data, sample 
preparation protocol, and various other relevant data.  The PDB is the main 
repository for structural data.  BMRB depositions may be linked to PDB
depositions \cite{bmrb, pdb}.




% figures
%\clearpage
%\section{Figures}

