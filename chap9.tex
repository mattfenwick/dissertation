\chapter{Conclusions}

\begin{center}
  \textit{Dealing with failure is easy: Work hard to improve. Success is also 
    easy to handle: You've solved the wrong problem. Work hard to improve.}

 - Alan Perlis
\end{center}


\section{The future of NMR as an experimental technique}
NMR is an important technique for studying biological molecules.
It is useful for structural, binding, and metabolomics studies.
However, to continue to grow in usefulness, several pressing problems must be solved.
These problems are varied in nature: some are inherent, and caused by the 
experimental phenomenon of studying nuclei in large magnets using 
radiofrequency pulses; others find their roots in our approach to 
understanding and analyzing the data.

The ability to isolate pure, high-concentration and stable samples of proteins 
is a prerequisite to carrying out an NMR study.  However, this is not only
difficult, but may be impossible in certain cases given our current techniques
\cite{bellstedt2013resonance}.  Worse still is that even for proteins which
can be studied effectively, it is difficult to ascertain how relevant the 
information obtained is to the protein's actual structure and function in the
complex biological system in which we desire to understand its role.  Thus, 
the challenge facing NMR is how to expand its reach to study additional proteins,
and to study proteins in native conditions.

Large proteins also pose severe problems for effective NMR data collection.  
This is because as molecular size increase, peak widths generally increase as
well; at the same time, the number of peaks increases because there are more
atoms in the molecule.  The result is decreased resolution, increased overlap,
and a decreased signal-noise ratio.  All of these lead to data that is more
difficult or impossible to interpret.  One current approach is to chop large
proteins into several smaller pieces which can be studied independently; the
hope is that the results gained are relevant to the full molecule.  
Progress is also being made in the area of improved pulse sequence design
and labeling schemes \cite{tzeng2012nmr}.  Nevertheless, the problem of molecule
size remains an ongoing challenge facing the field of NMR.

Effectively passing on knowledge to new students, and efficiently such that
they are able to quickly make valuable contributions is a problem of a different
sort.  While resources such as \url{http://www.protein-nmr.org.uk/} and books
such as \cite{hoch1996nmr} and \cite{keeler2013} are incredibly valuable and
helpful tools for learning, the task of learning NMR not only in breadth and
in depth, but also from a practical standpoint, remains a formidable and
time-consuming one.  In my opinion, a major contributor to this problem is 
the means employed for transmission of knowledge, information, and data of
the analysis of specific proteins and biological systems:
a portion of the information is implicitly and transiently communicated, such 
that the recorded analysis lacks appropriate context.
Books, websites, data archives, and journal publications are clearly effective
in explicitly transmitting and disseminating information; just as clearly,
these are not intended to provide complete information and data for specific
analyses.  Thus, the problem facing
NMR is how to identify and provide the resources necessary for efficiently and
effectively introducing new persons into the field. 

Related to previous problem is fostering an understanding and a means of 
discussing and sharing flaws in our work.  I believe the ability to correctly
recognize and understand problems is a prerequisite for solving them; and that
making progress in the field requires such recognition and solutions.  
Furthermore, I believe that in a collaborative community in which the work of
one individual becomes the basis for the work of another, such as the current
scientific community in which we work, explicitly identifying problems and
open issues is as important as making new discoveries; the cycle of first
identifying, and then solving problems is natural to science.  Therefore, 
the field of NMR must determine how to improve its communication of flaws 
and holes in our research, in order to facilitate the solution of them.


\section{Reproducibility: challenge, opportunity}
The major contribution of this work is substantial progress towards solving
the latter two problems by means of reproducibility.  I have presented 
approaches, models, and applications and techniques of those, all aimed toward
two goals: first, to make information and its context explicit, by recording
and preserving them; and second, to create a concept, means, and vocabulary for
identifying and communicating issues of NMR data analysis.  While these 
accomplishments only directly help to solve the latter two problems, I believe
they will also help to deal with the former two as well -- reproducibility
leads to improved and robust data analysis, which is a prerequisite for dealing
with the lower quality data collected from large or unstable proteins.


\section{The future of NMR software}
Software is integrated into every phase of the NMR analysis process.
Good software facilitates progress, while bad software restricts it, and so
software plays a major role in determining the quality, speed, and 
robustness of NMR analysis.  Two measures of software quality are how well a 
software product meets its users' current needs, and how well it is able to
adapt to meet their future needs.  Easily adapting to meet new 
challenges enables fast and cheap innovation; conversely, high
barriers to adapting and building software restrict innovation and prevent
improvements.  While NMR software has achieved incredible results, I believe
it is reaching a crossroads with respect to its adaptability.

If both flexibility and adaptability are to be achieved,
a new approach to building scientific software must be adopted and 
a reevaluation of the costs and value of software to NMR must be made 
in order to capitalize on software's potential.  
These changes are fundamentally different from rewriting software
to be more efficient, or have fewer bugs or more features.  Rather, the
goals must be to create software such that it is simple, obvious, and 
composable.  Such design goals naturally lead to correct, feature-rich, 
adaptable, and maintainable software -- but the converse is not true.  Worse
is that complex, non-composable software, instead of enabling
us, places an upper limit on the problems that can be solved using it.

CONNJUR is an open source project which provides flexible, composable tools
through data integration.  While it is not expected to solve every software
problem NMR spectroscopists face, I believe the software created by the CONNJUR
project embodies the principles needed to create effective scientific
software, through its bottom-up design and open source licensing:
the first is the best way we know of for dealing flexibly with changing
requirements, and the second is the best way we know of to ensure the 
continuous development and availability of valuable scientific software
projects.


\section{Final thoughts}
In this current climate of decreased scientific funding levels, and increased
competition for precious grant dollars, it is important to consider the effect
of science on society, and the effect of society on science.
While good science can have a positive impact on individuals and on society 
as a whole, bad science can have a corresponding negative impact.
I therefore hope that the goal of reproducibility is taken seriously by all
scientists, not only as a means of expanding human knowledge more quickly and
with less effort, but also a means to minimize the harm of bad science.
I look forward to science of the highest quality.

