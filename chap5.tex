\chapter{Software for Practical, Reproducible Analysis}

\section{Sparky extension}

\section{NMR-Star library}
NMR-Star is the file format used by the BMRB \cite{bmrb} for archival of
NMR data.  As such data is useful for further studies, and archiving
data in the BMRB is the primary means of dissemination, it is important
to be able to work with NMR-Star files.

This library facilitates reproducibility by providing a robust interface
for working with NMR-Star files.  It allows both the creation of NMR-Star 
files as well as extraction of data -- for further querying -- of existing
files.  It is used by the previously mentioned Sparky extension.

To handle these files, a library was implemented both in Java 
\cite{fenwick2013} and in Python.  This library provides capabilities both
for reading and for writing NMR-Star files.  Although several tools for
dealing with NMR-Star files had already been implemented \cite{ccpn, bmrb},
there are several attributes of this library which set it apart:
\begin{itemize}
  \item error reporting of illegal input.  When malformed input is encountered,
    a useful, location-specific error is reported which includes sufficient
    information to quickly pinpoint and diagnose the problem.
  \item complete, standards-compliant NMR-Star syntax definition.
  \item open source under the MIT license.  This allows other interested 
    developers to peruse the source code to gain ideas, use the library in
    new applications, and modify and extend the library to fix problems or
    add new features if necessary.
  \item low coupling.  As a simple library, in order to use it, the library
    is simply imported using standard language facilities in order to use
    it through its programmatic API.  It does not require any external tools
    or dependencies, reducing the barrier to setup and installation.  It does
    not require learning to use additional tools or languages, merely the 
    host language; it takes advantage of the native facilities for abstraction
    and composition provided by the host language.
  \item high cohesion.  The library provides a simple, focused interface.
    This means it is easy to learn and use because it only deals with parsing
    the concrete syntax of NMR-Star files.
\end{itemize}
The library is freely available online (
\url{https://pypi.python.org/pypi/NMRPyStar}, 
\url{https://github.com/CONNJUR/StarParser}
).

\section{Connjur: ST and WB}

\section{Sample Scheduler}
The creation of effective sample schedules is an important aspect of efficient,
non-uniform data collection 
\cite{maciejewski2011random, rovnyak2004accelerated, mobli2010non}.  There
are multiple strategies for data collection.  The strategy used to generate
a sample schedule and the exact sample schedule used to collect time-domain
data has an effect on the quality of the data and on the ease of later 
analysis, due to properties such as artifacts (described by the point-spread
function), resolution, and sensitivity (related to signal-to-noise ratio).

A tool has been implemented to reproducibly capture the parameterizations
used in sample schedule creation, and is available online at
\url{https://github.com/mattfenwick/PyScheduler}.  The tool features a 
collection of popular algorithms for creating non-uniform sample schedules
with specific, desirable properties.  The algorithms are integrated within
a single uniform, consistent interface which allows all input parameters
and outputs to be captured and archived.

The tool also features a data model for formal, precise communication of
sample schedules.  The data model not only deals with non-uniform time delays,
but also with non-uniform quadrature \cite{maciejewski2011random} and
non-uniform numbers of transients.  To my knowledge, the latter aspect of
non-uniform data collection remains a relatively unexplored domain, into
which this tool provides novel data representation capabilities.


\section{Discussion and conclusions}

