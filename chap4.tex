\chapter{Reproducible NMR data sets}

In order to prove the validity of the approach described in the previous
chapter, it was applied to typical NMR data in order to solve a protein
structure.  The process was carried out in full, and sufficient data was
captured during the process in order to render it reproducible.


\section{Methods and Materials}
 - CCPN Analysis
 - Sparky
 - Python
 - git
 - data model and approach from chapter 3
 - samp3 time-domain data from Mark
 - Cyana
 - NMRPipe
 - NMR-Star library
 - NMR-Star data dictionary


\section{NMR-Star data set}


\section{Difficulties encountered}


\section{Alternative implementations}

While the approach taken was to create the data set as a single NMR-Star
file for reasons of compatibility with existing programs and data archival
and retrieval facilities, there are other solutions to this problem.  This
section will inspect the strengths and weaknesses of an anlternative solution
which was taken during preliminary stages of this project.

The approach centered around using the open source VCS tool git 
\cite{loeliger2012git}.  A VCS tool enables the history of filesystem trees,
typically source code trees of software projects, to be captured in a series
of successive snapshots.  Git, as a distributed VCS, includes a more robust
model of snapshot history, which was critical in the success of its 
application to NMR data analysis.

Using git, multiple snapshots of a JSON or NMR-Star flat text file were
taken during the analysis process.  Each snapshot was annotated with a 
JSON formatted string which provided the deductive reasoning behind the change,
as well as a timestamp, author, and parent commit.  This approach was initially
employed due to git's intended use aligning very well with the domain problem.

The solution was quite easy, flexible, and robust in practice.  Git is a mature,
popular tool and therefore it was easy to learn to use, and contains many
features for creating, tracking and querying the history.  Here is a code 
snippet that shows how to create and annotate a new snapshot using git; all
commands are run from a standard shell:
\begin{verbatim}
# prepare two files -- peaks and parameters -- for snapshotting
$ git add nhsqc_peaks.str peak_picker_parameters.str

# create a snapshot, and annotate with a structured reason
$ git commit -m "{\"reason": "automated Sparky peak picker\"}"

# display the history of commits in the project
$ git log
\end{verbatim}

The git approach was the inspiration for the solution eventually employed.
There is an inherent tradeoff between the two approaches: in the git approach,
the full data set is spread across multiple snapshots of files, and the meta
data is also separate from the primary data.  In the NMR-Star based approach,
all the data, primary and meta, is in a single file.  While the git approach
makes it easier to query a single snapshot in time, it is more difficult to
query across multiple snapshots, or meta data.  The single file approach makes
it more difficult to query a single snapshot (other than the most recent one),
but easier to query across the complete data set including meta data.  It is
important to note that these differences are merely incidental, and not 
fundamental -- the same data is represented and stored in both approaches, it
is merely writing specific queries that becomes somewhat harder or easier.

Git includes additional features that were not taken advantage of in our
implementation, which could perhaps prove valuable at some later date.  An
important example is branches, which allow data analysis to fork and rejoin.
% TODO need a figure
This could potentially be useful for interpreting ambiguous or unclear features
in multiple ways, because it would allow an explicit record of the choice point.
It could also be useful for error correction, to show the cascading effect of
an incorrect analysis early in the process.


\section{Discussion}
 - richer queries now possible; let's do some


\clearpage
\section{Figures}

