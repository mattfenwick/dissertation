\chapter{An Approach for Reproducible Analysis}

\section{Background}
% maybe cover the current state of NMR reproducibility again?
% maybe talk about the culture of the lab notebook?


\section{Missing Primary Data}
need to cover what the data is, how it plays a role in NMR analysis, 
why it's important to capture

\subsection{Resonances and GSSs}
 - what: already covered in earlier section
 - how do we capture them: by adjusting the data model like CCPN, not going
   down the xeasy destructive-towards-spin-systems road, and use the BMRB's
   model for them when depositing
 - why they're so important to capture: they're a key intermediary between
   the experimental data and the atomic model

\subsection{Intermediates}
 - what: analysis data before analysis is complete
 - how it plays a role: shows the process of analysis -- from tool output to
   final results, as well as logical dependencies of derived data
 - why: allows recapitulation of the actual process employed for analysis,
   as well as fine-grained error detection and correction
   % ^^ oops, looks like I'm mixing in my solution along with the problem
 - the context of deductions is important: it determines what deductions
   can be made.  therefore capturing the contexts is important
 - I can come up with a simple example that demonstrates: 1) the dependence of
   deductive ability on context, 2) the concept and utility of multiple 
   snapshots, 3) the concept and utility of tracking logical dependencies
 - maybe an aspect of reproducibility is to make explicit what the variables,
   biases, input are.  so here, we're going to take the implicit context of
   deductions and make it explicit

\subsection{Extraneous Data}
 - what: explicit identification of data features as being 'not of interest'
   to this analysis
   - peaks
   - resonances
   - GSSs
 - how it plays a role: if they're not identified, you could get: wrong 
   chemical shifts, wrong sequential GSS GSS, wrong GSS residue assignments
 - why: provides a marker for future perusers
 - allows separation of identifying a data feature vs. interpreting it -- which
   matters because the way we do things is self-consistent, right?  
   so if you have to come back and reinterpret -- and remember, PINE is great and the
   lesson is that you *will* have to reinterpret -- you won't have to reidentify
   features.  
   and that's *good*, because that means that your feature identification
   can be unbiased.  
 - provides additional context for estimating the quality of an analysis,
   where errors may be likely to come from (the borderline cases!), helps with
   assigning confidence levels to datums by not forcing it to be a binary 
   decision.  quality measures: \# of peaks found by peak-picker, \# of false
   positives, \# of additional peaks picked, \# of peaks assigned to GSS, 
   \# of GSS assigned to residue, etc.
 - also, reporting these gives measures of, perhaps, contamination,
   incompleteness, overcompleteness/overfitting, consistency.  
 - remember another
   lesson from PINE: there's a balance between false positives and false negatives,
   and if you want to avoid false positives -- which I think we usually do -- 
   you'll probably have some false negatives.  so they should be reported
 - the fact that a peak was found, and later interpreted as noise does 
   not show up in the final data set.


\section{Missing Metadata}
 - what is the difference between primary data and meta data?  or maybe
   I should not use primary in that sense ....

\subsection{Deductive Analysis}
 - examples: what are they?  NMR-domain-specific rules for analyzing data
 - examples: how do we use them?
 - examples: how is recording them useful (to others)?
   provides explanation of why something was done -- which adds semantic data
   to a deduction; this semantic data provides additional meaning, which can
   be used for checking, learning, etc.

\subsection{Notes}
 - examples: what are they?
 - examples: how are they useful?  facilitates future reinterpretation if
   additional data is made available.  similar to PINE -- use notes to mark
   low-confidence portions of analysis
   allows highlighting of known flaws, as well as indicating how a data set
   may be improved


\section{A Model for Reproducible NMR}
 - what is a data model, and how is one used?
 - builds on BMRB and CCPN


\section{An Implementation of the Model}


\section{Applying Reproducible Analysis}
 - what to do -- tips and suggestions to be effective
 - what not to do -- common roadblocks and problems
% could this belong in its own chapter?
% or in the software chapter, or in the 'reproducible data sets' chapter?


\section{Archiving Reproducible Data Sets}
 - the NMR-Star file format, my parser, and others
 - the usefulness of NMR-Star files
 - the NMR-Star data dictionary
 - extending the NMR-Star data dic
 - other efforts to provide this NMR-Star integration ???


\section{Discussion}
 - lab notebooks
 - what data is missing
 - extending existing models to support these data
 - getting a handle on bias of data analysis


\section{Conclusion}
 - could go in with previous section

