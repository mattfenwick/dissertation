\chapter{Conclusions}

\section{The importance of open source scientific software}

Scientific software is often not shared.
This restricts progress and hinders reproducibility.
Releasing scientific software under open source licenses is one means to
address these problems. 

We want scientific software to be flexibile and adaptable, but also robust and consistent.
It is difficult to meet both these needs simultaneously.

One possible means to do so is known as "bottom-up" \cite{bottomup1994, bottomup2004} design
(as opposed to "top-down").  This emphasizes to solve small problems simply and 
completely, and then to build bigger software by combining the small solutions.  
Bottom-up solutions are easier to implement effectively when the solution is not known in advance, and are also better
equipped for dealing with changing requirements \cite{topdown_bottomup, bottomup1994, bottomup2004}.  

Such has been my experience.  The NMR-STAR library was designed to be bottom-up.
The result is that it is flexible and readily adapted to different uses.
Avoiding coupling \cite{coupling1992} was important in allowing this flexibility.


\section{The future of NMR}
There are several inherent problems for NMR (molecule size, obtaining pure and high-concentration samples).
There are also several incidental problems caused by irreproducibility.
Reproducibility allows us to focus on the inherent problems and make faster progress with them.
It is important to be able to effectively pass on knowledge to new students.
It is also important to be able to recognize, understand, and address the problems in the field of NMR.


\section{Final thoughts}

% where are we going? how will we get there?
% the dangers of irreproducibility

