\chapter{Conclusions}

\begin{center}
  \textit{Dealing with failure is easy: Work hard to improve. Success is also 
    easy to handle: You've solved the wrong problem. Work hard to improve.}

 - Alan Perlis
\end{center}

\section{The significance of NMR as an experimental technique}
While NMR has historically been an important technique for studying biological
molecules at an atomic level, it is neither the only technique for such 
studies, nor is it guaranteed to remain so indefinitely.  Although NMR does
currently enjoy a stranglehold in the area of dynamics and has proven itself
many times over as an effective technique for structural, binding, and metabolomics
studies, I believe that NMR's continued success rests on the ability of the
field to solve the several pressing problems facing it.

These problems are varied in nature: some are inherent, and caused by the 
experimental phenomenon of studying nuclei in large magnets using 
radiofrequency pulses; others find their roots in our approach to 
understanding and analyzing the data.

The ability to isolate pure, high-concentration and stable samples of proteins 
is a prerequisite to carrying out an NMR study.  However, this is not only
difficult, but may be impossible in certain cases given our current techniques
\cite{bellstedt2013resonance}.  Worse still is that even for proteins which
can be studied effectively, it is difficult to ascertain how relevant the 
information obtained is to the protein's actual structure and function in the
complex biological system in which we desire to understand its role.  Thus, 
the challenge facing NMR is how to expand its reach to study additional proteins,
and to study proteins in native conditions.

Large proteins also pose severe problems for effective NMR data collection.  
This is because as molecular size increase, peak widths generally increase as
well; at the same time, the number of peaks increases because there are more
atoms in the molecule.  The result is decreased resolution, increased overlap,
and a decreased signal-noise ratio.  All of these lead to data that is more
difficult or impossible to interpret.  One current approach is to chop large
proteins into several smaller pieces which can be studied independently; the
hope is that the results gained are relevant to the full molecule.  
Progress is also being made in the area of improved pulse sequence design
and labeling schemes \cite{tzeng2012nmr}.  Nevertheless, the problem of molecule
size remains an ongoing challenge facing the field of NMR.

Effectively passing on knowledge to new students, and efficiently such that
they are able to quickly make valuable contributions is a problem of a different
sort.  While resources such as \url{http://www.protein-nmr.org.uk/} and books
such as \cite{hoch1996nmr} and \cite{keeler2013} are incredibly valuable and
helpful tools for learning, the task of learning NMR not only in breadth and
in depth, but also from a practical standpoint, remains a formidable and
time-consuming one.  In my opinion, a major contributor to this problem is 
the means employed for transmission of knowledge, information, and data:
information is often implicitly and transiently communicated, such that it is
not understandable without the appropriate context, nor is it recorded.
Books, websites, data archives, and journal publications are clearly effective
in explicitly transmitting and disseminating information; just as clearly,
these and related media are not used for everything.  Thus, the problem facing
NMR is how to identify and provide the resources necessary for efficiently and
effectively introducing new persons into the field. 

Related to previous problem is fostering an understanding and a means of 
discussing and sharing flaws in our work.  I believe the ability to correctly
recognize and understand problems is a prerequisite for solving them; and that
making progress in the field requires such recognition and solutions.  
Furthermore, I believe that in a collaborative community in which the work of
one individual becomes the basis for the work of another, such as the current
scientific community in which we work, explicitly identifying problems and
open issues is as important as making new discoveries; the cycle of first
identifying, and then solving problems is natural to science.  Therefore, 
the field of NMR must determine how to improve its communication of flaws 
and holes in our research, in order to facilitate the solution of them.


\section{Reproducibility: challenge, opportunity}
 - solve the incidental problems, so that we can focus on the inherent ones
 - reproducibility shows where the problems are -- so that they can be identified and solved


\section{Software is indispensable}


\section{CONNJUR: open source, bottom-up data integration is a powerful approach}

