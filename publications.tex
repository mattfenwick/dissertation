\chapter{List of Publications}

\section{Reproducible protein NMR data analysis (in progress)}
This paper will cover the definition of reproducibility, the approach to
applying reproducibility to NMR data analysis, the library of deductive
reasons used for annotating NMR analysis, the Sparky extension which 
facilitates practical reproducibility, the extensions to the NMR-STAR
data dictionary to enable archiving and dissemination of reproducible
data sets, and a reproducible data set of a protein analyzed using the Sparky
extension according to the above approach, with the final data set in 
NMR-STAR format.  This is the subject of Chapters 1 through 7.


\section{CONNJUR Workflow Builder (in progress)}
This paper \cite{connjur-wb} will cover the design, applications, and context
of our new software product, WB, which is used for
spectral reconstruction.  This product is novel because it facilitates 
reproducible spectral reconstruction, metadata correctness, and is designed
to seamlessly deal with non-trivial workflows including branching and 
iteration, while interfacing with a relational database to provide centralized
data storage.  In addition, it is a valuable learning tool.  This is covered
in Chapters 3, 4, and 8.


\section{A bioinformatics sandbox}
In order to meet the growing need for scientific software that is 
simultaneously robust, flexible, and maintainable, this paper 
\cite{fenwick2012} applies an alternative approach, Functional Programming,
to developing scientific software.  This approach offers several potential
benefits which are explored using several small software projects. 
This paper also presents a public repository for programmers and bioinformaticians 
seeking to apply Functional Programming to biologically relevant problems; the
repository serves as a means to learn about Functional Programming and to share
interesting and useful code with other bioinformaticians.
This is covered in Chapter 8.


\section{Accessing archived NMR data}
This follow-up paper to the sandbox paper applies Functional Programming
to create \cite{fenwick2013} an application capable of reading data directly
from standard NMR archives.  This problem is difficult to solve, such
that most existing solutions are incomplete and difficult to use.  The solution
presented in this paper is complete and simple, in part due to the 
Functional Programming concepts applied when implementing the solution.
This is covered in Chapters 6, 7, and 8.


\section{Random phase detection}
While non-uniform sampling in time increments had previously been studied,
this paper \cite{maciejewski2011random} presents the additional concept and
characterization of non-uniformly sampling the phases as well.  This offers
potential reductions for spectrometer usage without sacrificing data quality.
This is covered in Chapters 3, 4, and 8.


\section{Software architecture for effective NMR data processing}
This paper \cite{connjur_pipeline} explores the architecture and design of 
CONNJUR software and how it contributes to the creation of a robust and
powerful framework for describing and solving NMR problems.
This is covered in Chapter 8.


\section{CONNJUR Spectrum Translator}
This paper \cite{connjur-st} describes the design and use of a tool for 
converting between several formats for binary time- and frequency-domain
NMR data.  ST provides a novel architecture for 
reducing the amount of work required to translate between formats.  This is
covered in Chapters 3, 4, and 8.

