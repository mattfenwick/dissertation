\chapter{Introduction}

NMR (Nuclear Magnetic Resonance) spectroscopy is an experimental technique for 
studying proteins and other biological molecules at atomic resolution.  In 
NMR is used to collect detailed molecular data, which 
is analyzed to get information about structure, dynamics,  
activity, and inter-molecular interactions.  
Since 1990, the structures of nearly 9,000 
proteins that were solved using NMR have been deposited in the Protein Data 
Bank (PDB) \cite{pdb}, a facility for the archival and sharing of protein-related 
data, and NMR data is available for over 10,000 proteins in the BioMagnetic 
Resonance Bank (BMRB) \cite{bmrb}, a facility for the archival and sharing of 
specifically NMR-derived data.  The data collected using NMR techniques is 
important to the field of drug design 
\cite{stockman2002drugs, moore2003leveraging, reckel2011proteorhodopsin}.

A major problem is that NMR analysis is irreproducible.  This document will
explain reproducibility and its significance, why NMR is irreproducible,
and present a strategy for rendering NMR analysis reproducible.



\section{Reproducibility}
Reproducibility is a critical component of science.
When the process for obtaining a result is captured in sufficient detail that
an independent scientist can execute the same process and get the same result,
it is said to be "reproducible" \cite{russell2013reproducibility, nih2014reproducibility}.
This allows it to be independently tested and verified.

There are several components of reproducibility: 
\begin{itemize}

  \item Experimental design.

  \item Experimental execution.

  \item Computational.  Indiscriminate computational use may be irreproducible \cite{donoho2009, peng2011reproducible}.  
    \cite{donoho_wavelab} provides a definition of computational reproducibility, and a strategy for achieving it.
    It is important that all tools, scripts, platforms and code are available, including the exact versions used
    \cite{ince2012open, nekrutenko2012next, blankenberg2014dissemination}.
    Parameters \cite{landis2012call}, input and output data must be captured.

  \item non-computational analysis.  failing to recognize analytical bias and to 
    use appropriate statistical measures has long been a source of 
    irreproducibility in scientific endeavors \cite{sackett1979bias}.
    See also \cite{ioannidis2005most, nuzzo2014statistical, begley2013reproducibility}, 
    and a recent Nature special (\url{http://www.nature.com/nature/focus/reproducibility/}).
    To ensure reproducibility of analysis, appropriate, necessary, and sufficient
    statistical measures should be used \cite{pashler2012replicability, vaux2012numbers}.
    If bias is present in the experimental data, its sources should be recognized
    and accounted for \cite{macarthur2012reproducibility, wagenmakers2012agenda}.

\end{itemize}



\section{Why is reproducibility important?}
Reproducibility is important to science for several reasons \cite{borgman2012conundrum}.
\begin{itemize}
  \item make it possible to judge the quality and value of a result
  \item facilitates the discovery and correction of bias
  \item facilitate transfer of knowledge and techniques between scientists \cite{rung2013reuse}
  \item facilitate collaboration
  \item facilitate error detection and correction
  \item facilitate teaching by explicitly capturing all relevant information,
        reducing information loss and reliance on implicits
  \item facilitate future reinterpretation and perusal of data and results
  \item higher quality results
  \item reduces time wastage due to inability to replicate results 
        \cite{ioannidis2005most, mullard2011reliability, prinz2011reproducibility, begley2012drug}. 
\end{itemize}


\section{NMR analysis is irreproducible}
Why?


\section{An Approach for Reproducible Analysis}
An approach for reproducibly analyzing NMR data was created and applied.

