\chapter{Introduction}


\section{What is reproducibility?}
Reproducibility is a critical component of science.
When the process for obtaining a result is captured in sufficient detail that
an independent scientist can execute the same process and get the same result,
it is said to be "reproducible" \cite{russell2013reproducibility, nih2014reproducibility}.
This allows it to be independently tested and verified.

There are several aspects of reproducibility: 
\begin{itemize}

  \item Experimental design.

  \item Experimental execution.  What was actually done, including reagents, conditions,
    mixing times, etc.

  \item computational analysis.  Indiscriminate computational use may be irreproducible \cite{donoho2009, peng2011reproducible}.  
    \cite{donoho_wavelab} provides a definition of computational reproducibility, and a strategy for achieving it.
    All tools, scripts, platforms and code must be available, including the exact versions used
    \cite{ince2012open, nekrutenko2012next, blankenberg2014dissemination}.
    Parameters \cite{landis2012call}, input and output data must be captured.

  \item non-computational analysis.  Analytical bias must be avoided \cite{sackett1979bias}.
    Appropriate, necessary, and sufficient statistical measures must be used 
    \cite{ioannidis2005most, nuzzo2014statistical, begley2013reproducibility,
          pashler2012replicability, vaux2012numbers}.
    Bias in the experimental data must be recognized
    and accounted for \cite{macarthur2012reproducibility, wagenmakers2012agenda}.

\end{itemize}



\section{Why is reproducibility important?}
Reproducibility is important to science for several reasons \cite{borgman2012conundrum}.
\begin{itemize}
  \item make it possible to judge the quality and value of a result
  \item facilitates the discovery and correction of bias
  \item enables transfer of knowledge and techniques between scientists \cite{rung2013reuse}
  \item promotes collaboration
  \item enables error detection and correction
  \item facilitates teaching by explicitly capturing all relevant information,
        reducing information loss and reliance on implicits
  \item makes future reinterpretation and perusal of data and results possible
  \item leads to higher quality results
  \item reduces time wastage due to inability to replicate results 
        \cite{ioannidis2005most, mullard2011reliability, prinz2011reproducibility, begley2012drug}. 
\end{itemize}



\section{What is NMR?}
NMR (Nuclear Magnetic Resonance) spectroscopy is an experimental technique for 
studying proteins and other biological molecules at atomic resolution.  In 
NMR is used to collect detailed molecular data, which 
is analyzed to get information about structure, dynamics,  
activity, and inter-molecular interactions.  
Since 1990, the structures of nearly 9,000 
proteins that were solved using NMR have been deposited in the Protein Data 
Bank (PDB) \cite{pdb}, a facility for the archival and sharing of protein-related 
data, and NMR data is available for over 10,000 proteins in the BioMagnetic 
Resonance Bank (BMRB) \cite{bmrb}, a facility for the archival and sharing of 
specifically NMR-derived data.  The data collected using NMR techniques is 
important to the field of drug design 
\cite{stockman2002drugs, moore2003leveraging, reckel2011proteorhodopsin}.



\section{How do we do NMR data analysis?}
NMR analysis is performed using a long, time-consuming process.  Many steps
are performed with the aid of computer programs, but most (if not all) require
some level of human intervention to verify and correct some portion of the
results.



\section{What NMR data is currently shared?}
Typically, only the final assigned chemical shifts are shared.  Additional 
supporting data including spin systems and resonances are not shared, nor is
the process of analysis itself, including what tools were used, their parameters,
and any manual modifications made.



\section{NMR analysis is irreproducible}
NMR analysis is irreproducible because the analysis process is not captured
in full.  For an overview of the analysis process, see \cite{guerry2011automated,
guntert2009automated, williamson2009automated}.  Briefly, a sequential process
is used, which may include the use of computational tools; manual modifications
are usually included as well, in order to validate, correct, and extend
computational analysis \cite{guerry2011automated, guntert2009automated, 
williamson2009automated}.  The manual interventions are necessary to get high-quality
results.

What necessary information is not captured from the analysis process?
\begin{itemize}
  \item what was done -- the tools that were used, their inputs and outputs,
    manual interventions
  
  \item the state of the analysis at each point during the process.
  
  \item why changes were made
  
  \item possible alternative interpretations
  
  \item issues with analysis -- problems, concerns, ambiguities, uncertainties,
    and incompletions
\end{itemize}



\section{Summary: problem and solution}
Reproducibility is important to all scientifici disciplines.  However, NMR
analysis is not reproducible.  The work presented here shows how NMR analysis
is irreproducible, defines a strategy for achieving reproducibility of analysis,
implements the strategy concretely, and applies the strategy to analyze real
NMR data, resulting in a reproducible data set.  The result is significant 
progress towards reproducibility.

