\chapter{Library of Deductive Reasons}
\label{sec_library}

This appendix presents commonly used deductive reasons during the process
of NMR data analysis.  It is grouped by the datatype being deduced, and then
broken down into the data used to make that deduction; each deduction is 
followed by an explanation of its meaning and applicability.


\section*{Pick new peak}

\subsection*{Local extremum of spectral intensity}
Peaks can be naively picked by identifying local maxima and minima in spectra.

\subsection*{Chemical shift matching}
One or more peaks assigned to a GSS may be used to pick a new peak,
based on matching one or more chemical shifts of the existing peaks
as well as spectral features.



\section*{Initialize new GSSs}

\subsection*{NHSQC peak}
Use NHSQC signal peaks to initialize GSSs.  This is a convenient choice due
to the many through-bond experiments which build upon the NHSQC's H-N coupling.



\section*{Peak classification as signal, noise, or artifact}

\subsection*{Peak intensity + noise level}
Signal peaks generally have high intensities and noise peaks have low ones.

\subsection*{Peak position + pulse sequence}
True signals are expected in limited spectral regions.
Projections of true signal peaks along spectral edges may indicate artifacts.

\subsection*{Peak sign + pulse sequence}
In experiments where all true peaks are expected to have the same sign, such 
as an NHSQC, a peak of the opposite sign may be a signal or an artifact.

\subsection*{Chemical shift matching + peak-GSS + pulse sequence}
A specific number of peaks are expected for a GSS in a given spectrum.
Extra peaks matching a GSS may be artifacts or noise.

\subsection*{Peak pattern}
Phase errors and truncations give rise to characteristic dispersive lineshapes
and sinc wiggles, respectively.  When using a naive peak picker, these may
be picked as a series of smaller peaks radiating out from a more intense,
central peak.

\subsection*{Chemical shift matching + pulse sequence + lack of peak}
A potential signal peak may be recognized as noise or artifactual based on
missing of matching peaks in another or the same spectra that would be 
expected if it were a true peak.



\section*{Assign peak cross section to resonance}

\subsection*{Peak cross section lineshape}
Peak cross sections from the same atom may have similar lineshapes;
if they are grouped into the same GSS, the matching cross sections can be
assigned to the same resonance.

\subsection*{Chemical shift matching}
The assignment of a peak cross section to a resonance can be used to assign 
another peak cross section from the same GSS to the same resonance, if the 
chemical shifts match.



\section*{Assign peak to GSS}

\subsection*{Peak-GSS: disambiguate peak-GSS assignment}
When peaks overlap, it is difficult to assign them correctly to GSSs.  However,
the resonances may be resolvable in additional spectra, which then makes it
possible to return to the first spectrum and correctly resolve its overlap.

\subsection*{Chemical shift matching}
Based on matching of corresponding peak cross sections, peaks are combined into
GSSs.  The matching of peaks may be within a single spectrum, or between 
multiple spectra -- as long as the spectral dimensions match (both nucleus
and atomtype).  The tolerances allowed are important in determining which
peaks match.



\section*{GSS typing}

\subsection*{Peak pattern}
In the C(CO)NH-TOCSY and HC(CO)NH-TOCSY, GSSs of specific types consistently
exhibit characteristic patterns of peak intensity and sign.  Observation of
a characteristic pattern can be used to deduce the type of a previously
untyped GSS.

\subsection*{BMRB statistics: sidechain Trp, Asn/Gln, Arg}
Sidechain Tryptophan GSSs exhibit a characteristic peak pattern with 
characteristic chemical shifts, which may be found in the BMRB.
Asparagine and Glutamine produce characteristic peak patterns in H-N-based 
experiments, due to their two sidechain protons.  These resonances consistently
appear in the same spectral region, making them easy to identify.
Arginine sidechain GSSs resonate at characteristic chemical shifts, and may
even be subject to splitting if they appear sufficiently far outside of the
decoupling band.

\subsection*{Resonance typing}
The atomtypes assigned to the resonances of a GSS place constraints on the
GSS's type.  For example, a GSS with a CB(i) resonance cannot be assigned to 
Glycine.

\subsection*{BMRB statistics: backbone Ala, Gly, Ser/Thr}
Alanine's CB resonates at a characteristic chemical shift relative to other 
CB's due to its lack of additional Carbon atoms.
Glycine's CA resonates at a characteristic chemical shift relative to other
CA's, and also lacks a CB atom.  In experiments such as an HNCACB or 
C(CO)NH-TOCSY, Glycine strips appear without a CB peak.
Serine's and Threonine's CB resonates at a characteristic chemical shift 
relative to other CB's due to the -OH groups.



\section*{Assign sequential GSS chain}

\subsection*{Chemical shift matching: GSS-GSS and resonance typing}
\label{subsec_shift_matching_seq_gss}
Matching Carbon strips from 3-dimensional experiments are used to build 
sequential GSS assignments and resolve some resonance types simultaneously. 
For example, given two HNCACB strips, chemical shift matching, and relative 
intensities, the sequential GSS assignments and CA(i)/CA(i-1) and CB(i)/CB(i-1) 
resonance typing assignments can be made, such that the following conditions
are satisfied (note that the last two are not inviolable):
\begin{itemize}
  \item i-1 peaks in following ss should be matched by i peaks in preceding ss
  \item intensity of HNCACB i-1 peaks should usually be less than intensity of i peaks in same ss
  \item intensity of HNCACB i-1 peaks should usually be less than intensity of matching i peaks in preceding ss
\end{itemize}

\subsection*{Resonance matching}
If the resonances are already typed (perhaps with the help of an auxiliary
experiment such as a C(CO)NH-TOCSY or HN(CO)CACB in addition to the HNCACB), 
sequential GSS assignment can be accomplished by matching resonances with
appropriate atomtype assignments: for example, a GSS with a CA(i) and CB(i)
resonance whose chemical shifts match the CA(i-1) and CB(i-1) resonance
frequencies can be deduced to form a sequential chain.



\section*{Sequence-specific GSS-residue assignment}

\subsection*{GSS typing + primary sequence}
A GSS chain can be assigned to residues based on the amino acid types of the 
residues and their match to the GSS typings.  It is not necessary that every
single GSS is unambiguously typed, merely that each GSS typing is consistent
with the primary sequence.

\subsection*{GSS typing + primary sequence + GSS-residue}
This is similar to the previous deduction, but takes previous sequence-specific
assignments into account using the process of elimination: for example, if a
GSS is typed as an Arginine, and there is only one unassigned Arginine residue 
remaining, the GSS may correspond to the Arginine.

\subsection*{Extend GSS-residue fragment}
A GSS chain already assigned to specific residues may be extended at the ends.
In addition to the constraints rules given in 
"chemical shift matching: GSS-GSS and resonance typing",
GSS typings must match the residue typing, or the BMRB statistics if untyped.



\section*{Resonance typing}

\subsection*{Peak sign + pulse sequence}
In experiments such as the HNCACB, the peak sign can contain information 
about the resonance typing in the 13C axis: CA resonances from backbone
GSSs have opposite signs from the CB resonances.

\subsection*{Resonance typing}
By process of elimination, if one resonance is assigned to a specific atomtype,
another resonance from the same GSS can not be assigned to the same atomtype.

\subsection*{Peak sign + GSS type}
The GSS type and peak sign can be used in conjunction with certain pulse 
sequences to determine resonance types.  For example, in an HNCACB spectrum, 
the positive peak of a Glutamine sidechain is a CG and the negative is a CB 
(or vice versa).

\subsection*{Sequential GSS assignment}
Resonance typing may be done during sequential GSS assignments, as covered
in section \ref{subsec_shift_matching_seq_gss}.

\subsection*{Characteristic splitting pattern}
Methylene groups, such as CB/HB2/HB3 in many amino acids, often show 
splitting in the 1H axis due to the two non-equivalent protons.

\subsection*{Pulse sequence}
The pulse sequence places constraints on the possible resonance typings of the
resonances of each peak: there is a limited list of spin systems which the
pulse sequence is capable of capturing; each peak from the spectrum must have
resonance typings corresponding to one of the possible spin systems.
For example, hbCBcgcdHD experiment targets aromatics; a peak from that spectrum
that matches a GSS not typed as an aromatic, can not be assigned 
to that GSS.

\subsection*{GSS typing}
A GSS's assigned type places constraints on the types of its resonances.
For example, a resonance in a Glycine GSS cannot be assigned to CB(i).

\subsection*{BMRB statistics}
The atomtype of resonances can often be assigned unambiguously using BMRB
statistics.  Process of elimination may also be employed.

\subsection*{BMRB statistics}
The BMRB average chemical shift stats -- along with the sstype assignment -- 
are useful for spectra such as the C(CO)NH-TOCSY, HBHA(CO)NH, and 
HC(CO)NH-TOCSY, where there are multiple peaks in a strip along a Carbon or
Hydrogen dimension. Most amino acid types feature good dispersion, making it 
easy to get the correct assignments straight directly from the statistics.

\subsection*{TOCSY aliphatic sidechain}
After assigning HNCACB and HBHA(CO)NH spectra, C(CO)NH-TOCSY, HC(CO)NH-TOCSY 
and HCCH-TOCSY are used in conjunction:
the C(CO)NH-TOCSY peaks are used to find HCCH-TOCSY strips, which yields proton 
chemical shifts and match the HC(CO)NH-TOCSY peaks.  Also, each HCCH-TOCSY 
strip should have peaks in all the same 1H shifts.
BMRB statistics can be used to assign most peaktypes unambiguously.  
Splitting patterns also help to identify methylene groups.

\subsection*{TOCSY peak pattern}
Several aliphatic C/H groups are difficult to distinguish using BMRB statistics,
including Leucine's CG, CD1, and CD2, as well as Isoleucine's CG2 and CD1, and
its QG2 and QD1. These can often be resolved by characteristic intensity 
patterns for each amino acid type due to the TOCSY nature of experiment,
as well as the observation that methyl peaks are often sharper and more intense.

\subsection*{CYANA: stereospecific resonance typing}
Many pairs of atoms/groups give rise to two peaks which can not be immediately
assigned unambiguously, although it is known that each atom/group corresponds 
to one of the peaks, and the other atom/group to the other peak.
Examples include HB2/HB3 of Y, and QD1/QD2 of L.
For additional examples of ambiguities see \ref{stereospecific_ambiguities}.
CYANA can often resolve these ambiguities during structure calculation.

