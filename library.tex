\chapter{Library of deductive reasons}
\label{sec_library}

\section{peak position}
Certain peak position can indicate whether a peak is signal, noise, or an 
artifact.  These positions include spectral regions that are expected to be 
empty, and projections of intensity along spectral edges.

\section{peak intensity}
Peaks from the same atom may have similar intensities. 
Signal, noise, and artifacts may have characteristic intensities.
This helps when assigning peak cross sections to resonances within a GSS.

\section{peak lineshape}
Peaks from the same atom may have similar lineshapes.
Signal, noise, and artifacts may have characteristic lineshapes.
This helps when assigning peak cross sections to resonances within a GSS.
 
\section{peak pattern}
Phase errors and truncations give rise to characteristic dispersive lineshapes
and sinc wiggles, respectively.  When using a naive peak picker, these may
be picked as a series of smaller peaks radiating out from a more intense,
central peak.  This helps to determine whether peaks are signal, noise, 
or artifacts.

\section{sign of peak amplitude}
In some experiments, all true peaks are expected to have the same sign.  In
other experiments, such as the HNCACB, some peaks are expected to consistently
have a single sign, and another set of peaks are expected to consistently 
exhibit the opposite sign.  Therefore, the sign can help to assign atomtype
to resonance, as well as to distinguish between signal and noise/artifact
in certain spectra.

\section{chemical shift matching: assign peaks to GSS}
Based on matching of corresponding peak cross sections, peaks are combined into
GSSs.  The matching of peaks may be within a single spectrum, or between 
multiple spectra -- as long as the spectral dimensions match (both nucleus
and atomtype).  The tolerances allowed are important in determining which
peaks match.

\section{chemical shift matching: recognize noise and artifacts}
A potential signal peak may be recognized as noise or artifactual based on
missing of matching peaks in other or the same spectra that would be expected
if it were a true peak.

\section{chemical shift matching: pick new peaks}
One or more peaks assigned to a GSS may be used to peak a new peak missed by a
peak picker, based on matching one or more chemical shifts of the existing peaks
as well as spectral features.

\section{chemical shift matching: assign resonance to peak cross section}
The assignment of a peak cross section to a resonance can be used to assign another
peak cross section from the same GSS to the same resonance, if the chemical shifts
match.

\section{BMRB statistics: resonance typing}
The atomtype of resonances can often be assigned unambiguously using BMRB
statistics.  Process of elimination may also be employed.

\section{GSS typing: sidechain Tryptophan}
Sidechain Tryptophan GSSs exhibit a characteristic peak pattern with 
characteristic chemical shifts, which may be found in the BMRB.

\section{GSS typing: sidechain Asparagine/Glutamine}
Asparagine and Glutamine produce characteristic peak patterns in H-N-based 
experiments, due to their two sidechain protons.  These resonances consistently
appear in the same spectral region, making them easy to identify.

\section{GSS typing: sidechain Arginine}
Arginine sidechain GSSs resonate at characteristic chemical shifts, and may
even be subject to splitting if they appear sufficiently far outside of the
decoupling band.

\section{resolve overlap: disambiguate peak-GSS assignment}
When peaks overlap, it is difficult to assign them correctly to GSSs.  However,
the resonances may be resolvable in additional spectra, which then makes it
possible to return to the first spectrum and correctly resolve its overlap.

\section{peak sign and GSS type: resonance typing}
The GSS type and peak sign can be used in conjunction with certain pulse 
sequences to determine resonance types.  For example, in an HNCACB spectrum, 
the positive peak of a Glutamine sidechain is a CG and the negative is a CB 
(or vice versa).

\section{experiment and GSS type: resonance typing and GSS-peak assignment}
This is similar to the previous reason.  
For example, hbCBcgcdHD experiment targets aromatics; a peak from that spectrum
that matches a GSS not typed as an aromatic, can not be assigned 
to that GSS.

\section{GSS typing: backbone Alanine}
Alanine's CB resonates at a characteristic chemical shift relative to other 
CB's due to its lack of additional Carbon atoms.

\section{GSS typing: backbone Gly}
Glycine's CA resonates at a characteristic chemical shift relative to other
CA's, and also lacks a CB atom.  In experiments such as an HNCACB or C(CO)NH-Tocsy,
Glycine strips appear without a CB peak.

\section{GSS typing: backbone Ser/Thr}
Serine's and Threonine's CB resonates at a characteristic chemical shift relative
to other CB's due to the -OH groups.

\section{chemical shift matching: GSS-GSS and resonance typing}
Matching Carbon strips from 3-dimensional experiments are used to build 
sequential GSS assignments and resolve some resonance types simultaneously. 
For example, given two HNCACB strips, chemical shift matching, and relative 
intensities, the sequential GSS assignments and CA(i)/CA(i-1) and CB(i)/CB(i-1) 
resonance typing assignments can be made, such that the following conditions
are satisfied (note that the last two are not inviolable):
\begin{itemize}
  \item i-1 peaks in following ss should be matched by i peaks in preceding ss
  \item intensity of HNCACB i-1 peaks should usually be less than intensity of i peaks in same ss
  \item intensity of HNCACB i-1 peaks should usually be less than intensity of matching i peaks in preceding ss
\end{itemize}

\section{unexpected peak: signal vs. noise/artifact}
A specific number of peaks are expected for a GSS in a given spectrum.
Extra peaks -- beyond the expected number -- may be artifacts or noise.

\section{GSS chain and primary sequence: GSS-residue assignment}
A GSS chain can be assigned to residues based on the amino acid types of the 
residues and their match to the GSS typings.  It is not necessary that every
single GSS is unambiguously typed.
The process of elimination is useful in this deduction; for example, if the
GSS is an Arginine, and there's only one unassigned Arginine residue remaining,
the GSS may correspond to the Arginine.

\section{extend GSS-residue fragment: GSS-residue assignment}
A GSS chain already assigned to specific residues may be extended at the ends.
In addition to the constraints rules given in 
"chemical shift matching: GSS-GSS and resonance typing",
GSS typings must match the residue typing, or the BMRB statistics if untyped.

\section{BMRB statistics: resonance typing}
The BMRB average chemical shift stats -- along with the sstype assignment -- 
are useful for spectra such as the C(CO)NH-Tocsy, HBHA(CO)NH, and 
HC(CO)NH-Tocsy, where there are multiple peaks in a strip along a Carbon or
Hydrogen dimension. Most amino acid types feature good dispersion, making it 
easy to get the correct assignments straight directly from the statistics.

\section{Tocsy aliphatic sidechain: resonance typing}
After assigning HNCACB and HBHA(CO)NH spectra, C(CO)NH-Tocsy, HC(CO)NH-Tocsy 
and HCCH-Tocsy are used in conjunction:
the C(CO)NH-Tocsy peaks are used to find HCCH-Tocsy strips, which yields proton 
chemical shifts and match the HC(CO)NH-Tocsy peaks.  Also, each HCCH-Tocsy 
strip should have peaks in all the same 1H shifts.
BMRB statistics can be used to assign most peaktypes unambiguously.  
Splitting patterns also help to identify methylene groups.

\section{Tocsy peak pattern: resonance typing}
Several aliphatic C/H groups are difficult to distinguish using BMRB statistics,
including Leucine's CG, CD1, and CD2, as well as Isoleucine's CG2 and CD1, and
its QG2 and QD1. These can often be resolved by characteristic intensity 
patterns for each amino acid type due to the Tocsy nature of experiment,
as well as the observation that methyl peaks are often sharper and more intense.

\section{Cyana: stereospecific resonance typing}
Many pairs of atoms/groups give rise to two peaks which can not be immediately
assigned unambiguously, although it is known that each atom/group corresponds 
to one of the peaks, and the other atom/group to the other peak.
Examples include HB2/HB3 of Y, and QD1/QD2 of L.
For additional examples of ambiguities see \ref{stereospecific_ambiguities}.
Cyana can often resolve these ambiguities during structure calculation.

